%!TEX root = forallxyyc.tex
\part{Noções-chave da lógica}
\label{ch.intro}
\addtocontents{toc}{\protect\mbox{}\protect\hrulefill\par}


\chapter{Argumentos}
\label{s:Arguments}

%O assunto da lógica é a avaliação de argumentos; a identificação dos bons argumentos, separando-os dos maus.  

%Na linguagem do dia a dia, às vezes usamos a palavra `argumento' para falar de bate-bocas e desacordos verbais.
%Se você e um amigo discutem nesse sentido, as coisas não estão indo bem entre vocês dois.
%Não é esse o sentido da palavra `argumento' no âmbito da lógica. Esses bate-bocas não são argumentos no sentido lógico. São apenas desacordos.

%Um argumento, no sentido em que empregaremos aqui, é algo mais parecido com isto:

O assunto da lógica é a avaliação de argumentos; a identificação dos bons argumentos, separando-os dos maus.
Um \define{argument} é um grupo de sentenças que exprime um raciocínio.
Uma das sentenças, chamada de \define{conclusion}, é o ponto final do raciocínio, e corresponde àquilo que se conclui.
As outras, chamadas de \definepl{premise}, são o ponto inicial do raciocínio e representam as informações das quais se parte para chegar à conclusão. 
Um argumento, então, é algo como:

	\begin{earg}\label{argButlerGardner}
		\item[] Ou foi o mordomo, ou foi o jardineiro.
		\item[] Não foi o mordomo.
		\item[\therefore] Foi o jardineiro.
	\end{earg}
	
%Aqui temos uma série de sentenças.
%Os três pontos na terceira linha do argumento significam ``portanto''.
%Eles indicam que a sentença final expressa a \emph{conclusão} do argumento.
%As duas sentenças anteriores são as \emph{premissas} do argumento.
%Se você acredita nas premissas e acha que a conclusão se segue das premissas---que o argumento, como diremos, é válido---então parece que você tem uma boa razão para acreditar na conclusão.

%É nesse tipo de coisa que os lógicos estão interessados. Diremos que um argumento é qualquer coleção de premissas, juntamente com uma conclusão.

\noindent Os três pontos na terceira linha significam ``portanto'' e indicam que esta sentença é a conclusão do argumento. As duas sentenças anteriores são as premissas.
Um argumento é considerado bom do ponto de vista lógico quando o raciocínio que parte das premissas e chega à conclusão for infalível, ou seja, quando acreditar que as premissas são verdadeiras nos dá razão suficiente para acreditar que a conclusão também é verdadeira. Sempre que isso ocorre dizemos que a conclusão se segue das premissas, ou que é uma consequência das premissas, e que o argumento é válido.
É nesse tipo de coisa que os lógicos estão interessados; em identificar e caracterizar os argumentos válidos, ajudando-nos com isso a reconhecer os raciocínios infalíveis.  

Os principais métodos que os lógicos desenvolveram para identificar os arugmentos válidos são métodos formais.
Eles envolvem uma espécie de tradução das sentenças que compõem os argumentos do português para certas linguagens simbólicas muito parecidas com a linguagem da matemática.

Estudaremos algumas destas linguagens e métodos em detalhes no decorrer deste livro, mas antes disso vamos neste e nos próximos dois capítulos que compõem a Parte I, introduzir os principais conceitos da lógica de um modo informal.
A ideia é obtermos um entendimento prévio mais ou menos intuitivo das principais noções lógicas, antes de as estudarmos formalmente.

É fundamental, então, começar com uma compreensão clara do que é um argumento, saber identificar quais são suas premissas e qual é sua conclusão, e entender o que  significa um argumento ser válido.


%A Parte I deste livro discute algumas noções lógicas básicas que se aplicam a argumentos em um idioma natural, tal como o português.
%É fundamental começar com uma compreensão clara do que são argumentos e do que significa um argumento ser válido.
%Mais tarde, traduziremos os argumentos do português para uma linguagem formal. Queremos que a validade formal, conforme será definida na linguagem formal, tenha pelo menos algumas das características importantes que a validade das linguagens naturais tem.

No exemplo recém-apresentado, expressamos cada premissa através de uma sentença separada, e usamos uma terceira sentença para  expressar a conclusão do argumento.
%Muitos argumentos são expressos dessa maneira, mas uma única sentença também pode conter um argumento completo.
Apesar de muitos argumentos serem expressos dessa maneira, pode ocorrer também de uma única sentença conter um argumento completo. 
Considere:
	\begin{quote}
		O mordomo tem um álibi, logo não foi ele.
	\end{quote}
%Este argumento tem uma premissa seguida de uma conclusão.
Identificamos um argumento nesta sentença, porque identificamos nela um raciocínio. Parte-se da premissa de que o mordomo tem um álibi e conclui-se que não foi ele.

Muitos argumentos começam com as premissas e terminam com uma conclusão, mas nem todos.
O argumento inicial desta seção poderia igualmente ter sido apresentado com a conclusão no início, da seguinte forma:
	\begin{quote}
		Foi o jardineiro. Afinal, ou foi o mordomo, ou o jardineiro. E não foi o mordomo.
	\end{quote}
O mesmo argumento também poderia ter sido apresentado com a conclusão no meio:
	\begin{quote}
		Não foi o mordomo. Consequentemente foi o jardineiro, dado que ou foi o mordomo, ou o jardineiro.
	\end{quote}
O importante aqui é percebermos que um texto qualquer será um argumento sempre que ele expressar um raciocínio, onde chega-se a uma conclusão a partir de uma ou mais premissas. 

Para avaliar um argumento e saber se o raciocínio que ele expressa é infalível, ou seja, se sua conclusão se segue ou não de suas premissas, a primeira coisa a fazer é identificar a conclusão e separá-la das premissas.
%Então, a primeira coisa a fazer é identificar a conclusão e separá-la das premissas.
As expressões do quadro abaixo são frequentemente usadas para indicar a conclusão de um argumento e são, por isso, chamadas de \definepl{conclusion indicator word}.
\factoidbox{
	\begin{description}
		\item[Expressões indicativas de conclusão:]
			\begin{itemize}
				\item[]
				\item logo
				\item portanto
				\item por conseguinte
				\item sendo assim
				\item assim
				\item deste modo
				\item por isso
				\item em vista disso
				\item isto (prova/mostra/demonstra) que
				\item desta forma
				\item consequentemente
			\end{itemize}
	\end{description}
	}
Por outro lado, as expressões do quadro abaixo são \definepl{premise indicator word}, dado que geralmente elas indicam que a frase que as segue é uma premissa:
\factoidbox{
	\begin{description}
		\item[Expressões indicativas de premissa:]
			\begin{itemize}
				\item[]
				\item porque
				\item visto que
				\item desde que
				\item dado que
				\item uma vez que
				\item afinal
				\item afinal de contas
				\item pois
				\item assumindo que
				\item é sabido que
				\item por causa de
			\end{itemize}
	\end{description}
	}



%As expressões abaixo são frequentemente usadas para indicar a conclusão de um argumento:
%	\begin{center}
%		logo\\ portanto\\ por conseguinte\\ sendo assim\\ assim\\ deste modo\\ por isso\\ em vista disso\\ isto (prova/mostra/demonstra) que\\ desta forma\\ consequentemente
%	\end{center}
%Por esse motivo são, às vezes, chamadas de \definepl{conclusion indicator word}.

%Por outro lado, as expressões abaixo são \definepl{premise indicator word}, dado que geralmente elas indicam que a frase que as segue é uma premissa e não uma conclusão:
%	\begin{center}
%		porque\\ visto que\\ desde que\\ dado que\\ uma vez que\\ afinal\\ afinal de contas\\ pois\\ assuma que\\ é sabido que\\ por causa de
%	\end{center}
As expressões indicativas de conclusão tanto quanto as indicativas de premissa apenas nos dão uma pista.
O fundamental é ter em mente que a conclusão é sempre aquilo que se quer dizer, a ``moral da história'' que corresponde ao ponto final do raciocínio; já as premissas sempre são uma justificativa ou explicação da conclusão, são as informações das quais se parte para chegar à conclusão. 

\section{Sentenças}
\label{intro.sentences}

%De um modo bastante geral, podemos definir um \define{argumento} como uma série de sentenças.
%Uma delas, geralmente a última, é a conclusão, e as outras são as premissas.
%Se as premissas são verdadeiras e o argumento é bom, então você tem um motivo para aceitar a conclusão.

%Na lógica, estamos interessados apenas em sentenças que podem figurar como premissas ou conclusões de um argumento, ou seja, sentenças que podem ser verdadeiras ou falsas.
%Portanto, nos restringiremos a sentenças desse tipo e definiremos \define{sentence} como frases ou expressões que podem ser verdadeiras ou falsas.

Dissemos antes que os argumentos, foco central do interesse da lógica, são conjuntos de sentenças que exprimem raciocínios.
Mas raciocinar é manipular informação.
Por isso à lógica interessam apenas as sentenças capazes de carregar informação, que são exatamente aquelas que podem ser verdadeiras ou falsas.
Assumiremos, então, uma definição de \define{sentence} que contempla apenas aquelas frases ou expressões que podem ser verdadeiras ou falsas.

%Não confunda a ideia de uma sentença que pode ser verdadeira ou falsa com a diferença entre fato e opinião.
Essas sentenças, na maioria das vezes, expressam coisas que contam como \textit{fatos}, tais como ``Kierkegaard era corcunda'' ou ``Kierkegaard gostava de amêndoas''.
Mas as sentenças da lógica também podem expressar coisas que nos parecem mais uma \textit{opinião} do que um fato, tais como ``Amêndoas são saborosas''.
Estas expressões de opinião também são sentenças legítimas, no sentido lógico que estamos adotando aqui.
Em outras palavras, uma sentença não é desqualificada como parte legítima de um argumento só porque não sabemos se ela é verdadeira ou falsa, nem porque sua verdade ou falsidade é uma questão de opinião.
Não importa se sabemos, nem mesmo se é possível ou não saber se a sentença é verdadeira ou falsa.
Se a sentença for do tipo que pode ser verdadeira ou falsa, então ela será uma sentença em nossa acepção lógica  e pode desempenhar o papel de premissa ou conclusão e fazer parte de um argumento lógico.
Essas sentenças, sobre as quais cabe julgar se são verdadeiras ou falsas, são conhecidas como \emph{sentenças declarativas}.

Por outro lado, há expressões que seriam consideradas sentenças por um linguista ou gramático, mas que não são sentenças declarativas e portanto, não contam como sentenças na lógica, tais como:

\paragraph{Perguntas \ --}Em uma aula de gramática, a expressão `Você já está com sono?'  contaria como uma sentença interrogativa. Mas ainda que você esteja mesmo sonolento, a pergunta em si não será verdadeira por causa disso.
Perguntas, em geral, não fazem declarações e por isso não são nem verdadeiras nem falsas e não contam como sentenças na lógica.
Elas não podem fazer parte de um argumento nem como premissas nem como conclusões.
Se, por exemplo, você disser ``não estou com sono'' em resposta à pergunta acima, sua resposta será uma sentença no sentido lógico, porque diferentemente da pergunta, ela é do tipo que pode ser verdadeira ou falsa.
Geralmente, \emph{perguntas} não contam como sentenças, mas \emph{respostas} contam.

`Sobre o que é este curso?' não é uma sentença (no nosso sentido).
Por outro lado, `Ninguém sabe sobre o que este curso trata' é uma sentença.

\paragraph{Imperativos \ --}As ordens costumam ser formuladas como imperativos tais como ``Acorde!'', ``Sente-se direito'' e assim por diante.
Em uma aula de gramática, essas expressões contariam como sentenças imperativas.
Mas ainda que seja aconselhável sentar-se com a coluna ereta, a ordem não será verdadeira ou falsa por causa disso.
Observe, no entanto, que as ordens ou comandos nem sempre são expressos como imperativos.
Por exemplo, a expressão `Você respeitará minha autoridade' \emph{é} ou verdadeira ou falsa, pois você respeitará ou não.
Então, estritamente falando, trata-se de uma sentença no sentido lógico, ainda que consigamos perceber que por trás desta declaração há uma intenção de dar uma ordem.

\paragraph{Exclamações \ --}Expressões como `Ai!' às vezes são chamadas de sentenças exclamatórias.
No entanto, elas não são nem verdadeiras nem falsas.
No que diz respeito à lógica, vamos tratar aqui sentenças do tipo `Ai, machuquei meu dedão!' como significando a mesma coisa que `Machuquei meu dedão.'
O `ai' não acrescenta nada que possa alterar a verdade ou falsidade da sentença e, por isso, é desconsiderado nas avaliações lógicas.


\practiceproblems
No final da maioria dos capítulos há uma seção com exercícios que ajudam a revisar e explorar o material abordado no capítulo.
Fazer estes exercícios é parte essencial e insubstituível do seu aprendizado.
Aprender lógica é como aprender a falar uma língua estrangeira, ou aprender a jogar tênis, ou a tocar piano.
Não basta ler e entender a teoria.
A parte mais importante do aprendizado é a prática.

\medskip


Então, aqui está o primeiro exercício. Identifique a conclusão em cada um dos 4 argumentos abaixo.
\begin{earg}
	\item Faz sol. Logo eu deveria levar meus óculos escuros.
	\item Deve ter feito muito sol. Afinal de contas, eu estava de óculos escuros.
	\item Ninguém, exceto você, pôs as mãos no pote de biscoitos.
	E a cena do crime está cheia de migalhas de biscoito.
	Você é o culpado!
	\item A Srta. Rosa e o Prof. Black estavam no escritório na hora do crime.
	O Sr. Marinho estava com o candelabro no salão de festas,
	e sabemos que não há sangue em suas mãos.
	Consequentemente, o Coronel Mostarda cometeu o crime na cozinha, com a chave inglesa.
	Lembre-se de que a pistola não foi disparada.
\end{earg}



\chapter{O alcance da lógica}
\label{s:Valid}

\section{Validade e contraexemplo}
No Capítulo \ref{s:Arguments}, falamos sobre argumentos, ou seja, uma coleção de sentenças (as premissas), seguidas por uma única sentença (a conclusão).
Dissemos que algumas palavras, tal como ``portanto'', indicam qual sentença deve ser a conclusão.
A palavra ``portanto'', é claro, sugere que há uma conexão entre as premissas e a conclusão.
A ideia é que um argumento seja a expressão de um raciocínio cujo ponto de partida são as premissas e o ponto de chegada é a conclusão.
Por isso dizemos que a conclusão \emph{segue-se} ou é uma \emph{consequência} das premissas.

A principal preocupação da lógica é exatamente esta noção de consequência.
A lógica enquanto disciplina constitui-se em um conjunto de teorias e ferramentas que nos apontarão quando uma sentença se segue de outras, ou seja, quando o raciocínio que parte das premissas e chega na conclusão é um raciocínio infalível.

Pois bem, voltemos ao argumento apresentado no Capítulo \ref{s:Arguments}: 

	\begin{earg}
		\item[] Ou foi o mordomo, ou foi o jardineiro.
		\item[] Não foi o mordomo.
		\item[\therefore] Foi o jardineiro.
	\end{earg}
Não sabemos a que, exatamente, essas sentenças se referem.
Talvez você suspeite que ``foi'' signifique ``foi o autor de algum crime'' não especificado.
Podemos imaginar, por exemplo, que esse argumento tenha sido dito por um detetive que estivesse considerando as evidências de um crime em um livro de mistério ou em uma série de TV.

Mas mesmo sem saber a que exatamente a palavra ``foi'' se refere, você provavelmente concorda que o argumento é bom no sentido de que  se as premissas forem ambas verdadeiras, a conclusão não pode deixar de ser verdadeira também.
Se a primeira premissa for verdadeira, ou seja, se for verdade que ``ou foi o mordomo, ou foi o jardineiro'', então pelo menos um deles ``foi''.
E se a segunda premissa também for verdadeira, então não ``foi'' o mordomo.
Isso deixa apenas uma opção: ``foi o jardineiro'', que é exatamente a conclusão do argumento.
Então, acreditar que as premissas são verdadeiras nos dá um motivo incontestável para acreditar que a conclusão também deve ser verdadeira.
Um argumento que possui essa propriedade é chamado de \define{valid}.
Sua conclusão é \define{consequence} (ou se segue) das premissas.

Um argumento, então, é \emph{válido}, quando em qualquer situação possível na qual suas premissas são verdadeiras, sua conclusão também é.
Seria irracional, ilógico, acreditar nas premissas e não acreditar na conclusão de um argumento válido.
Nós sabemos que o argumento do mordomo e do jardineiro é válido porque mesmo sem saber exatamente do que as sentenças estão falando, sabemos que as premissas são uma garantia para a conclusão:
em qualquer situação na qual as duas premissas forem verdadeiras, a conclusão também será.

É esta ideia de garantia infalível, de justificativa confiável, que está por trás da definição de argumento válido.
Quando não existe situação possível em que as premissas são verdadeiras e a conclusão é falsa, então a verdade das premissas garante (justifica) a verdade da conclusão e o argumento é válido.

Veja mais dois exemplos de argumentos válidos.
\begin{earg}\label{ValidoMaria}
	\item[] Se Maria da Paz mora em Paris, então ela mora na França.
	\item[] Maria da Paz mora em Paris.
	\item[\therefore] Maria da Paz mora na França.
\end{earg}

\begin{earg}
	\item[] Todos os filósofos são repolhos.
	\item[] Ricardo é um filósofo.
	\item[\therefore] Ricardo é um repolho.
\end{earg}
Repare que em nenhum desses dois argumentos a conclusão é verdadeira.\footnote{
	Maria da Paz e Ricardo são dois dos muitos coautores deste livro. Ela mora em Natal e ele é um ser humano.}
E mesmo assim eles são válidos.
Isso pode ocorrer porque a validade de um argumento não nos diz nada sobre se sua conclusão é verdadeira ou não. 
Ela nos diz apenas que as premissas dão garantia para a conclusão.
O seja:
\begin{description}\label{premissafalsa}
	\item[Quando as premissas \textit{são} todas verdadeiras:] a validade do argumento garante que a conclusão também é.
	\item[Quando as premissas \textit{não são} todas verdadeiras:] a conclusão pode ser verdadeira ou falsa.
	A validade, nesse caso, garante apenas que se as premissas \textit{fossem} todas verdadeiras, a conclusão também \textit{seria}.
\end{description}
Então, quando alguma premissa é falsa, é perfeitamente aceitável que um argumento válido tenha conclusão falsa.
Considere agora o seguinte argumento:
\begin{earg}\label{argMaidDriver}
	\item[] Se foi o motorista, então não foi a babá.
	\item[] Não foi a babá.
	\item[\therefore] Foi o motorista.
\end{earg}
Continuamos sem saber exatamente o que está sendo dito aqui, já que não sabemos a que a palavra ``foi'' se refere.
No entanto, você provavelmente concorda que esse argumento é diferente dos anteriores em um aspecto importante.
Mesmo que suas premissas sejam ambas verdadeiras, não é garantido que a conclusão também será.
Ou seja, a verdade das premissas não é garantia para a verdade da conclusão.
Nós podemos ver isso, porque conseguimos conceber uma situação hipotética na qual as premissas do argumento são verdadeiras e a conclusão é falsa.
Considere:

\begin{itemize}\label{contraexemplo}
	\item ``foi'' significa ``foi a última pessoa a deixar a mansão na noite de ontem'';
	\item  a última pessoa que deixou a mansão ontem foi o jardineiro.
\end {itemize}
A primeira premissa de nosso argumento é verdadeira nessa situação, porque como a última pessoa a deixar a mansão é uma pessoa só, é claro que ``se foi o motorista, então não foi a babá''.
A segunda premissa também é verdadeira, porque na situação proposta o último a deixar a mansão ontem ``não foi a babá'' mesmo, mas o jardineiro.
Esse é também o motivo pelo qual a  conclusão é falsa.
O último a deixar a mansão não ``foi o motorista'', como afirma a conclusão, mas o jardineiro.

Chamamos de  \define{counterexample} uma situação como esta, em que as premissas de um argumento são todas verdadeiras, mas a conclusão é falsa.

Um \textit{contraexemplo} ilustra que a verdade das premissas não é garantia para a verdade da conclusão, já que nele as premissas são verdadeiras mas a conclusão não é.
Ele mostra explicitamente  uma falha no raciocínio que o argumento expressa.
Chamaremos de \define{invalid} qualquer argumento que tenha um contraexemplo.

Veja abaixo mais dois exemplos de argumentos inválidos. Tente imaginar contraexemplos para eles, ou seja, para cada um deles imagine uma situação possível na qual as premissas são verdadeiras e a conclusão é falsa.
\begin{earg}
	\item[] Se Maria da Paz mora em Paris, então ela mora na França.
	\item[] Maria da Paz mora na França.
	\item[\therefore] Maria da Paz mora em Paris.
\end{earg}

\begin{earg}
	\item[] Todos os filósofos são repolhos.
	\item[] Ricardo é um repolho.
	\item[\therefore] Ricardo é filósofo.
\end{earg}
Conforme já foi dito, a tarefa principal da lógica é classificar os argumentos, separando os bons, chamados válidos, dos maus, os inválidos.
E conforme acabamos de ver, a noção de contraexemplo fornece um modo preciso de realizar esta classificação. 
Argumentos válidos não têm contraexemplo, e argumentos inválidos têm.
Este será o principal critério que guiará nossos estudos nas primeiras partes deste livro.
Mas como você já deve ter notado, temos usado muitas outras expressões diferentes relacionadas às noções de validade e invalidade de argumentos.
E temos feito isso porque há mesmo muitas maneiras alternativas pelas quais podemos entender a ideia de validade, o que pode nos deixar um pouco confusos.

Para evitar confusão, vamos registrar aqui as várias definições que costumam ser empregadas.
O primeiro quadro abaixo contém dez diferentes modos equivalentes de dizer que um argumento é válido e o segundo quandro contém dez diferentes modos equivalentes de dizer que um argumento é inválido.
Ou seja, no contexto deste livro, todas as dez sentenças do primeiro quadro dizem a mesma coisa: o argumento é válido.
E todas as dez sentenças do segundo quadro também dizem a mesma coisa: o argumento é inválido.

\factoidbox{
\begin {itemize}
	\item[] \textbf{Diferentes modos de dizer a validade:}
	\begin{enumerate}\small
		\item O argumento é válido.
		\item O argumento não tem contraexemplo.
		\item Não existe situação possível em que todas as premissas são verdadeiras e a conclusão é falsa.
		\item Se as premissas forem verdadeiras, é garantido que a conclusão também será.
		\item O raciocínio que parte das premissas e chega à conclusão é infalível.
		\item A conclusão se segue das premissas.
		\item A conclusão é consequência das premissas.
		\item As premissas justificam a conclusão.
		\item A informação da conclusão está contida na informação das premissas.
		\item O argumento é bom.
	\end{enumerate}
\end{itemize}
}
\ \
\factoidbox{
\begin{itemize}	
	\item[] \textbf{Diferentes modos de dizer a invalidade:}
	\begin{enumerate}\small
		\item O argumento é inválido.
		\item O argumento tem contraexemplo.
		\item Existe pelo menos uma situação possível na qual todas as premissas são verdadeiras e a conclusão é falsa.
		\item Mesmo se as premissas forem todas verdadeiras, não é garantido que a conclusão também será.
		\item O raciocínio que parte das premissas e chega à conclusão não é infalível e o contraexemplo ilustra um caso em que ele falha.
		\item A conclusão não se segue das premissas. 
		\item A conclusão não é consequência das premissas.
		\item As premissas não justificam a conclusão.
		\item A informação da conclusão não está contida na informação das premissas.
		\item O argumento é mau.
	\end{enumerate}
\end{itemize}
}


\section{Situações}
\label{ss:Validade}
Dissemos que o que distingue os argumentos válidos dos inválidos é que os primeiros não têm contraexemplo e os últimos têm.
E dissemos também que um contraexemplo é uma situação possível na qual as premissas do argumento são verdadeiras e a conclusão é falsa. 
Mas há ainda um elemento um pouco misterioso aqui.
O que exatamente é uma situação possível?
Existem situações impossíveis?
E o que significa ser verdadeiro em uma situação?

Nosso primeiro esclarecimento é de vocabulário.
Estamos usando as expressões `situação' e `situação possível' como sinônimas.
Então, no nosso uso aqui neste livro, não existem situações impossíveis.
Situações são situações possíveis.
Mas afinal o que é uma situação?

Em linhas bem gerais uma \define{case} é um estado de coisas possível, ou seja, é um cenário hipotético, uma estória imaginada com informação suficiente para decidir se as sentenças de um argumento são verdadeiras ou falsas.
Para descrever uma situação, além de apresentar o cenário hipotético (a estória), precisamos também estabelecer o significado preciso de eventuais palavras ambíguas presentes nas sentenças.
Já vimos, na Página \pageref{contraexemplo}, um exemplo de situação. % o número da página aparece certo, mas o link leva para a primeira página da seção onde a referência ocorre
Tinhamos ali o seguinte argumento:
 
 \begin{earg}
 	\item[] Se foi o motorista, então não foi a babá.
 	\item[] Não foi a babá.
 	\item[\therefore] Foi o motorista.
 \end{earg}
E propuzemos a seguinte situação, que mostramos ser um contraexemplo para esse argumento.

\begin{description}
	\item[situação 1:] ``foi'' significa ``foi a última pessoa a deixar a mansão na noite de ontem''; e a última pessoa que deixou a mansão ontem foi o jardineiro.
\end {description}
Veja que esta situação é composta por (\textit{a}) uma especificação do significado da palavra ``foi'', que é ambígua sem um contexto; e por (\textit{b}) uma estória.
Com esses dois elementos, a situação configura-se em um cenário hipotético que tem informação suficiente para decidir a verdade ou falsidade das três sentenças do argumento.
Conforme vimos, as duas premissas são verdadeiras e a conclusão é falsa, o que faz desta situação um contraexemplo que comprova que o argumento é inválido.

Considere agora este outro argumento:
\begin{earg}
	\item[] Todos os natalenses são canhotos.
	\item[] Jordão é  natalense.
	\item[\therefore] Jordão é canhoto.
\end{earg}
Aqui uma situação é dada por qualquer estória onde podemos consultar se essas sentenças são verdadeiras ou falsas segundo a estória.
Vejamos um exemplo:

\begin{description}
	\item[situação 2:] há 150 anos caiu um estranho meteorito nos arredores do forte dos Reis Magos em Natal-RN, e desde então, não se sabe bem por que, todas as pessoas que nascem naquele município são canhotas. Jordão nasceu em Natal há 21 anos e é canhoto.
\end{description} 
Se usamos a estória da situação 2 como referência, todas as três sentenças do argumento acima são verdadeiras.
Dizemos então que elas são verdadeiras na situação 2.
Considere agora esta outra situação.

\begin{description}
	\item[situação 3:] Jordão é canhoto e nasceu em Natal há 21 anos.
\end{description} 
Repare que a estória contada nessa situação é minimalista.
Muito pouco é dito. Apenas que Jordão é natalense, canhoto e tem 21 anos.
Ou seja, está explícito na situação 3 que a segunda premissa e a conclusão são verdadeiras.
Mas e a primeira premissa?
A estória não nos diz nada explicitamente sobre se todos os natalenses são canhotos ou não.
Diz que Jordão é canhoto, mas e os outros natalenses?
Neste caso fazemos aqui o que fazemos quando ouvimos qualquer estória. O que não nos é dito, deve ser igual ao mundo real.
No mundo real há natalenses destros.
Então a primeira premissa é falsa na situação 3.
Sempre que a estória que descreve uma situação omite algo, usamos o mundo real para preencher a parte que falta da estória e completar a situação.

O resultado aqui é que na situação 3 a segunda premissa e a conclusão são verdadeiras, mas a primeira premissa falsa.
Então, assim como a 2, a situação 3 também não é um contraexemplo para o nosso argumento.

Mesmo que você seja a pessoa mais imaginativa do mundo e passe a vida tentando, sem trapacear\footnote{
	Trapacear aqui seria modificar o significado das palavras lógicas `todos' ou `é' (mais adiante explicaremos o que são `palavras lógicas'), ou então estipular que as outras palavras, como `natalense' ou `canhoto' ou mesmo `Jordão' significam coisas diferentes em sentenças diferentes.
	Isso é trapaça porque fazer isso não é encontrar uma estória que seja contraexemplo para o argumento, mas é mudar o significado das sentenças do argumento para que com significado novo elas se encaixem em alguma estória que não seria contraexemplo se você não tivesse mudado o significado das sentenças.}
você não vai conseguir inventar uma estória onde as premissas desse argumento são verdadeiras e a conclusão é falsa.
A informação da conclusão já está contida na informação das premissas.
Por isso esse argumento não tem contraexemplo e é válido.

Mas suponha que você, inconformado com isso, proponha a seguinte suposta situação, alegando que ela é um contraexemplo:
\begin{description}
	\item[suposta situação 4:] há 150 anos caiu um estranho meteorito nos arredores do forte dos Reis Magos em Natal-RN, e desde então, não se sabe bem por que, todas as pessoas que nascem naquele município são canhotas. Jordão nasceu em Natal há 21 anos e é destro.
\end{description} 
As duas premissas de nosso argumento parecem verdadeiras nessa suposta situação, já que a queda do meteorito assegura que todos os natalenses vivos hoje são canhotos e, além disso, é dito explicitamente que Jordão nasceu em Natal.

Mas com a conclusão temos um problema.
Por um lado é dito explicitamente que Jordão é destro, o que parece assegurar que a conclusão é falsa.
Mas por outro lado também é dito que Jordão nasceu em Natal há 21 anos e que todas as pessoas que nasceram em Natal nos últimos 150 anos são canhotas.
Então é claro que Jordão é uma dessas pessoas e por isso é canhoto.
Ou seja, a estória dessa suposta situação está dizendo que Jordão é destro e canhoto.
Como assim?
Jordão é canhoto ou destro?
Ele não pode ser as duas coisas.
Quem é canhoto não é destro e quem é destro não é canhoto.\footnote{
	Apenas para não deixar dúvida, estamos usando as palavras `destro' e `canhoto' com os seguintes sentidos: uma pessoa canhota tem mais habilidade com o lado esquerdo do corpo do que com o direito. Uma pessoa destra tem mais habilidade com o lado direito do corpo do que com o esquerdo. E se uma pessoa tem habilidades equivalentes com os lados esquerdo e dirieto, esta pessoa não é nem canhota nem destra, é ambidestra. Então se uma pessoa é canhota, ela não é destra. E se ela é destra, ela não é canhota.}

Se você pensar bem, verá que também há problemas com a primeira premissa, que diz que ``todos os natalenses são canhotos''.
Porque a estória diz que Jordão, um natalense de 21 anos,  é destro, o que torna a primeira premissa falsa.
Mas a estória também diz que todos que nasceram em Natal nos últimos 150 anos são canhotos, o que torna a premissa verdadeira.
Ou seja, a primeira premissa do argumento parece tanto verdadeira quanto falsa nessa estória.

O que devemos fazer diante de uma bagunça como essa?
Se  você tivesse ouvido alguém contar essa estória, depois de refletir e perceber esses problemas, tenho certeza que você concluiria que ou a pessoa que contou a estória está enganada, ou então que ela está mentindo.
Essa estória não pode ser real.

É isso mesmo o que fazemos na lógica.
A estória da suposta situação 4 não descreve um estado de coisas possível, um modo como o mundo poderia ser.
Não é possível que o mundo seja do jeito que essa estória descreve, porque de acordo com ela as sentenças `Jordão é canhoto' e `Todos os natalenses são canhotos' são tanto verdadeiras quanto falsas.
Essa estória contém o que chamamos de contradições e, por isso, extrapola os limites do que consideramos concebível para o mundo.

Uma situação, conforme definimos, precisa representar uma possibilidade legítima e concebível para o mundo, e a suposta situação 4 não faz isso.
Ao conter contradições ela viola as regras do que é possível e por isso não é uma situação legítima; não é uma história, mas apenas uma estória inconcebível e contraditória.
%Ela é mesmo só uma estória, e não uma história.

Isso nos deixa com algumas perguntas importantes.
Quais são essas regras do que é possível, que foram violadas pela suposta situação 4?
Como podemos saber o que é possível e o que não é possível?
Como se define os limites do que é concebível para o mundo?

Na situação 2, por exemplo, um meteorito caiu em Natal e como consequência inexplicável desse fenômeno todas as pessoas nascidas na cidade desde então são canhotas.
Isso é bem esquisito e absurdo, mas não rechaçamos a situação 2 como ilegítima ou inconcebível.
Ela foi considerada uma situação legítima.
Já a suposta situação 4, onde Jordão é canhoto e não é canhoto, e é destro e não é destro foi recusada como ilegítima e inconcebível.
Por que é concebível que a queda de um meteorito faça com que todas as pessoas que venham a nascer nas proximidades sejam canhotas, mas é inconcebível que uma pessoa seja e não seja canhota?

Não temos respostas para todas estas perguntas.
Na verdade não há respostas unânimes para todas estas perguntas.
Elas têm sido respondidas diferentemente por pesquisadores diferentes.
O que você precisa saber disso tudo é que podemos entender a lógica como a disciplina que estuda exatamente estas regras do que é possível, do que consideramos concebível.
Ou seja, são as regras da lógica que estabelecerão os limites do que pode figurar nas situações.
Qualquer situação, por mais estranha que seja, que não viole princípios considerados lógicos, será aceita como legítima e concebível.
Serão inconcebíveis e inaceitáveis apenas aquelas supostas situações cujas estórias violem algum princípio lógico.

%A principal tarefa dos lógicos é tornar esta misteriosa noção de situação mais precisa e investigar em que medida diferentes modos de tornar mais precisa a noção de situação afetam quais argumentos serão classificados como válidos e quais não serão. 

A principal tarefa dos lógicos é explicitar as regras que definem os limites do que é aceitável nas situações possíveis e, com isso, tornar essa misteriosa noção de situação mais precisa.
Mas, como tudo em filosofia, não há limites, critérios ou padrões rígidos de como se deve fazer isso.
Por isso, também é tarefa dos lógicos investigar em que medida diferentes modos de tornar mais precisa a noção de situação afetam quais argumentos serão classificados como válidos e quais não serão.

\section{Tipos de validade}
%Podemos, por exemplo, nos perguntar:
Consdere, por exemplo, estas duas perguntas sobre os limites do que é aceitável nas situações:
\begin{itemize}
	\item As situações devem ser limitadas pelas leis científicas?
	\item São elas obrigadas a serem compatíveis com nossos conceitos e o modo como estes se relacionam uns com os outros?
\end{itemize}
Estas são perguntas sobre qual é o alcance e o limite da lógica e respostas diferentes a elas  levarão a diferentes modos de separar os argumentos em válidos e inválidos.

Suponha, por exemplo, que os cenários hipotéticos sejam limitados pelas leis da física.
Ou seja, suponha que uma situação que viola alguma lei da física não possa ser considerada como um motivo legítimo para refutar um argumento.
Considere, então, o seguinte argumento:
	\begin{earg}
		\item[] A espaçonave \emph{Rocinante} levou 6 horas na viagem entre a estação espacial Tycho e o planeta Júpiter.
		\item[\therefore] A distância entre a estação espacial Tycho e Júpiter é menor do que 14~bilhões de quilômetros.
	\end{earg}
Um contraexemplo para esse argumento seria um cenário hipotético em que a nave \emph{Rocinante} faz uma viagem de mais de 14 bilhões de quilômetros em 6 horas, excedendo, assim, a velocidade da luz.
Esse cenário é incompatível com as leis da física, já que de acordo com elas nada pode exceder a velocidade da luz.
Então, se aceitarmos a suposição de que as situações devem respeitar as leis da física, não conseguiremos produzir nenhum contraexemplo para esse argumento, que será, por isso, considerado válido.

Por outro lado, se os cenários hipotéticos puderem desafiar as leis da física, então é fácil propor um no qual a premissa desse argumento é verdadeira e a conclusão é falsa.
Basta que, nesse cenário, a nave \emph{Rocinante} viaje mais rápido que a luz.
Sendo aceitável, esse cenário torna-se um contraexemplo para o argumento que, por isso, não será considerado válido.

O que devemos fazer aqui?
Se por um lado não faz muito sentido admitir situações que violem as leis da natureza, já que parece que elas nunca ocorrerão, por outro lado, proibir tais situações pode nos limitar demais.
Afinal, nem sempre soubemos o que hoje sabemos.
Os conhecimentos que a ciência nos dá hoje, algum dia já foram cenários hipotéticos que violavam as leis científicas vigentes.
Os lógicos preferem deixar a ciência para os cientistas e aceitam como concebíveis situações que violam as leis da natureza.
Então, de um ponto de vista lógico, o argumento acima da nave \textit{Rocinante} é inválido, já que situações nas quais a nave viaja mais rápido do que a luz são aceitáveis e constituem contraexemplos para o argumento.

Se, no entanto, quisermos levar em consideração algumas das restrições da ciência em nossos raciocínios, sempre podemos acrescentar as leis científicas que nos interessam como premissas de nossos argumentos.
Por exemplo:
%Acrescentando a lei científica que afirma que nada ultrapassa a velocidade da luz como premissa
	\begin{earg}
		\item[] A espaçonave \emph{Rocinante} levou 6 horas na viagem entre a estação espacial Tycho e o planeta Júpiter.
		\item[] Nada viaja mais rápido do que a luz (300.000 quilômetros por segundo)
		\item[\therefore] A distância entre a estação espacial Tycho e Júpiter é menor do que 14~bilhões de quilômetros.
	\end{earg}
Temos agora um argumento válido, porque a segunda premissa proibe explicitamente situações que violem o limite da velocidade da luz.

Suponha, agora, que os cenários hipotéticos sejam limitados pelos nossos conceitos e pelo modo como eles se relacionam, e considere este outro argumento:
	\begin{earg}
		\item[] Jussara é uma oftalmologista.
		\item[\therefore] Jussara é uma médica de olhos.
	\end{earg}
Se estamos permitindo apenas cenários compatíveis com nossos conceitos e suas relações, então esse também é um argumento válido.
Afinal, em qualquer cenário que imaginarmos no qual Jussara é uma oftalmologista, Jussara será uma médica de olhos, já que os conceitos de ser uma oftalmologista e ser uma médica de olhos têm o mesmo significado.
Se, no entanto, permitimos situações em que ser uma oftalmologista significa algo diferente de ser uma médica de olhos, então é claro que a premissa desse argumento pode ser verdadeira e a conclsão falsa.

O que devemos fazer?
Faz sentido admitir situações que violam as relações entre os conceitos e os significados das palavras?
Ou seja, situações em que ser oftalmologista e ser médica de olhos não são a mesma coisa?
Se por um lado parece um pouco forçado distorcer o significado das palavras nos cenários hipotéticos que admitimos, por outro lado, as relações de conceitos e significados  não são tão rígidas quanto parecem.
Eles podem mudar com o tempo, podem ser diferentes de pessoa para pessoa, podem depender do contexto.

E, pelo menos idealmente, os lógicos pretendem que a validade dos argumentos não seja algo assim tão variável.
Lembre que a ideia é que os argumentos válidos representem raciocínios infalíveis.
Então, aqui também a lógica tomará a posição mais liberal e, de modo geral, admitirá situações que violam as relações conceituais e o significado das palavras.
Assim, de um ponto de vista lógico, o argumento acima sobre Jussara é inválido, porque são admissíveis situações em que os conceitos de ser oftalmologista e ser médica de olhos não são idênticos, o que possibilita a existência de contraexemplos para o argumento.

Novamente, sempre que quisermos raciocinar levando em consideração determinadas relações conceituais específicas, podemos acrescentar premissas em nossos argumentos.
Então, se quisermos proibir cenários nos quais ser uma oftalmologista e ser uma médica de olhos são coisas diferentes, podemos reescrever o argumento acima com uma premissa extra que garante isso e obtemos, assim, o seguinte argumento válido:

	\begin{earg}
		\item[] Jussara é uma oftalmologista.
		\item[] Alguém é oftalmologista se e somente se é médico de olhos.
		\item[\therefore] Jussara é uma médica de olhos.
	\end{earg}
Vamos chamar de \define{nomologicamente valido}\label{nomoval} um argumento para o qual não há contraexemplos que não violem as leis da natureza; e vamos chamar de \define{conceitualmente valido} um argumento para o qual não há contraexemplos que não violem as conexões de nossos conceitos.

Esses dois exemplos de validades alternativas, ambas diferentes da validade lógica mais tradicional que estudaremos neste livro, ilustram algo muito importante:
a noção de validade de argumentos depende dos limites que estamos dispostos a impor às situações aceitáveis.
Limites diferentes nos dão diferentes noções de validade e consequência.

%Isso mostra que a lógica e suas noções principais, tais como a validade e a consequência, não é anterior, separada ou prioritária em relação a outros domínios, tais como o domínio da realidade natural e das leis da natureza, e o domínio dos conceitos e significados das sentenças.
%O modo como entendermos e concebermos estes outros domínios poderá interferir e alterar nosso entendimento sobre se um argumento é válido ou não.


\section{Validade formal e o método da lógica}
A ciência é um empreendimento extremamente complexo, mutável, e algumas vezes até controverso. As relações entre conceitos e os significados que atribuímos às palavras são ainda menos seguros e mais disputáveis do que a ciência.
Ao liberar as situações possíveis de ter que respeitar as leis científicas e as relações tradicionais entre os conceitos, os lógicos não apenas simplificam seu trabalho, como também adicionam segurança e força à sua disciplina.

Quando a lógica reconhece um argumento como válido, ele é válido a despeito da ciência e dos significados das palavras.
Um argumento que a lógica reconhece como válido continuaria sendo válido mesmo se as leis científicas fossem totalmente diferentes do que elas são, e mesmo que as palavras tivesse significados totalmente diferentes dos que elas têm.\footnote{
	Na verdade, veremos mais adiante que a validade lógica depende do significado de umas poucas palavras (as palavras lógicas) que descrevem a forma lógica das sentenças.}
Vejamos um exemplo:
\begin{description}
\item[Argumento 1]
\begin{earg}
	\item[]
	\item[] Jussara é uma oftalmologista ou uma dentista.
	\item[] Jussara não é uma dentista.
	\item[\therefore] Jussara é uma oftalmologista.
\end{earg}
\end{description}
Esse argumento é conceitualmente válido.
Sem mudar os significados das palavras, você não vai conseguir encontrar contraexemplo para ele.
Mas a validade dele é ainda mais forte do que isso.
Mesmo que você altere os significados dos conceitos `oftalmologista' e `dentista' e a referência do nome `Jussara', ainda assim não vai conseguir encontrar contraexemplo para o argumento.
Ele é \define{formalmente valido}.
Sua validade não depende do conteúdo das senteças, mas apenas de sua forma lógica.
Podemos ver um esboço da forma lógica desse argumento através do seguinte padrão:

%Esse tipo de validade extremamente forte que a lógica busca é chamada de \textit{validade formal}.
%A ideia é que a validade formal não dependa do conteúdo das sentenças do argumento, mas apenas de sua \textit{forma lógica}.

\begin{description}
\item[Padrão 1]
\begin{earg}
	\item[]
	\item[] $A$ é um $X$ ou um $Y$.
	\item[] $A$ não é um $Y$.
	\item[\therefore] $A$ é um $X$.
\end{earg}
\end{description}
Aqui, `$A$', `$X$' e `$Y$' funcionam como espaços reservados para expressões apropriadas que, quando substituem `$A$', `$X$' e `$Y$', transformam esse padrão em um argumento constituído por sentenças em português.
Por exemplo,
\begin{description}
\item[Argumento 2]
\begin{earg}
	\item[]
	\item[] Edna é uma matemática ou uma bióloga.
	\item[] Edna não é uma bióloga.
	\item[\therefore] Edna é uma matemática.
\end{earg}
\end{description}
é um argumento cuja forma lógica também tem o padrão 1, onde substituímos `A' por `Edna', `X' por `matemática' e `Y' por `bióloga'.
Note que esse argumento também é válido.
Os argumentos 1, 2 e todos os outros cuja forma lógica tem o padrão 1 são válidos.
%É isso, no final das contas, que significa dizer que eles são formalmente válidos, que eles são válidos independentemente do conteúdo das sentenças, são válidos apenas em virtude de sua forma lógica.
Considere agora este outro argumento.

\begin{description}
\item[Argumento 3]
\begin{earg}
	\item[]
	\item[] Jussara é uma oftalmologista ou uma dentista.
	\item[] Jussara não é uma dentista.
	\item[\therefore] Jussara é uma médica de olhos.
\end{earg}
\end{description}
Ele é conceitualmente válido.
Nenhuma situação que respeite o fato de que `oftalmologista' e `médica de olhos' têm o mesmo significado será contraexemplo desse argumento.
Apesar disso esse argumento não é formalmente válido.
Um esboço de sua forma lógica é dado pelo seguinte padrão,
\begin{description}
\item[Padrão 2]
\begin{earg}
	\item[]
	\item[] $A$ é um $X$ ou um $Y$.
	\item[] $A$ não é um $Y$.
	\item[\therefore] $A$ é um $Z$.
\end{earg}
\end{description}
 onde apenas substituímos `Jussara' por `$A$',  `oftalmologista' por `$X$', `dentista' por `$Y$' e  `médica de olhos' por `$Z$'.
 Mas se agora tomamos esse padrão 2 e substituímos `$A$' por `Edna', `$X$' por `matemática', `$Y$' por `bióloga' e `$Z$' por `trapezista', obtemos o seguinte argumento:
\begin{description}
\item[Argumento 4]
\begin{earg}
	\item[]
	\item[] Edna é uma matemática ou uma bióloga.
	\item[] Edna não é uma bióloga.
	\item[\therefore] Edna é uma trapezista.
\end{earg}
\end{description}
Esse argumento é claramente inválido, uma vez que podemos imaginar uma situação em que uma matemática chamada Edna  não é  trapezista nem  bióloga.
Nesta situação as duas premissas do argumento 4 são verdadeiras, mas a conclusão não é.
Isso mostra que a forma lógica esboçada no padrão 2 não é uma forma lógica válida:
ela não garante a validade conceitual de todos os argumentos que a compartilham.

Esses exemplos nos dão um esboço rudimentar do método da lógica para o reconhecimento da validade formal dos argumentos.
Este método contém dois elementos fundamentais:
\begin{enumerate}
	\item  uma linguagem artificial especialmente projetada para exprimir apenas a forma lógica dos argumentos, desconsiderando os conteúdos de suas sentenças;
	\item um método que, dada a forma lógica de um argumento expressada nessa linguagem artificial, calcula se esta é uma forma lógica válida ou inválida.
\end{enumerate}
A ideia que fundamenta a validade formal e o método da lógica é a de que qualquer argumento que tenha uma forma lógica reconhecida como válida será um argumento válido.
Então, dados esses dois elementos, a avaliação lógica de um argumento segue os seguintes passos:
\begin{itemize}
	\item Em primeiro lugar obtemos a forma lógica do argumento que queremos avaliar traduzindo-o para uma linguagem artificial. Nos exemplos acima, o padrão 1 esboça a forma lógica dos argumentos 1 e 2, e o padrão 2 a forma lógica dos argumentos 3 e 4.
	\item Em seguida aplicamos o método de cálculo lógico que classifica a forma lógica como válida ou inválida. Não fizemos este cálculo nos exemplos acima, mas a ideia é que vamos ter métodos para dizer, por exemplo, que o padrão 1 esboça uma forma lógica válida, e o padrão 2 uma forma lógica inválida.
	\item O argumento original terá a mesma classificação de sua forma lógica. Será classificado como válido se sua forma lógica for válida, e inválido se sua forma lógica for inválida. Então, em nossos exemplos, os argumentos 1 e 2, que compartilham o padrão 1 válido, serão argumentos válidos, e os argumentos 3 e 4, que compartilham o padrão 2 inválido, serão argumentos inválidos.
\end{itemize}
Antes de prosseguir, é importante fazermos aqui um alerta vocabular sobre o uso das classificações válido, inválido e suas variações.
Apresentamos três tipos diferentes de validade: \textit{validade nomológica}, \textit{validade conceitual} e \textit{validade formal}.
Mas na maioria das vezes você não vai ouvir as pessoas qualificando a validade com esses adjetivos.
Elas vão dizer simplesmente que os argumentos são válidos ou inválidos, sem qualquer especificação.
Como saber de qual validade elas estão falando?
Não há uma regra rígida sobre isso, mas há uma tradição, que é a seguinte:
\begin{description}
	\item[válido = conceitualmente válido --] quando estamos refletindo sobre a validade ou invalidade de argumentos em um contexto informal, onde eles estão escritos em português e não vamos formalizá-los nem aplicar métodos específicos da lógica formal, neste contexto classificar um argumento como válido normalmente significa classificá-lo como conceitualmente válido.
	\item[válido = nomologicamente válido --] em alguns casos bastante específicos de avaliação informal de argumentos, principalmente no âmbito do discurso das ciências, a classificação `válido' significa `nomologicamente válido'.
	A distinção entre este caso e o anterior não é rígida.
	Se você não tiver certeza de que o contexto científico se aplica, adote o entendimento mais liberal do caso anterior e entenda válido como conceitualmente válido.
	\item[válido = formalmente válido --] quando estamos refletindo sobre a validade ou invalidade de argumentos em um contexto formal, técnico, auxiliados pelas ferramentas da lógica, a classificação de um argumento como válido significa que ele é formalmente válido.
	Neste livro, com exceção destes capítulos introdutórios da Parte I, será esta a interpretação que utilizaremos.
	Veremos, inclusive, que como há diversos sistemas lógicos diferentes, haverá também diversas validades formais diferentes.
\end{description}

%Mas se desconsiderarmos a equivalência entre esses dois conceitos, podemos propor, por exemplo, a seguinte situação:
%`oftalmologista' significa `samambaia' e `Jussara' é o nome que dei para a samambaia da minha varanda.
%Nesta situação, as duas premissas do argumento 3 são verdadeiras, mas a conclusão é falsa, porque a samambaia de minha varanda certamente não é uma médica de olhos.
%Então, apesar de conceitualmente válido, este argumento não é formalmente válido.

%Faremos algumas suposições sobre os diversos tipos diferentes de situações que admitiremos na análise da validade de um argumento.
%A primeira suposição é que toda situação admissível tem que ser capaz de determinar a verdade ou não de cada sentença do argumento em consideração.
%Isso significa, em primeiro lugar, que não será aceito como uma situação admissível para um possível contraexemplo, qualquer cenário imaginário no qual a verdade de alguma sentença do argumento considerado não seja determinada.
%Por exemplo, um cenário em que Jussara é dentista, mas não oftalmologista, contará como uma situação a ser considerada nos primeiros argumentos desta seção, mas não como uma situação a ser considerada nos últimos dois argumentos:
%este cenário nada nos diz sobre se Edna é matemática, bióloga ou trapezista.
%Se uma situação não determina que uma sentença é verdadeira, diremos que ela determina que a %sentença é \define{false}.
%Assumiremos, então, que as situações determinam a verdade ou a falsidade das sentenças, mas nunca ambas.\footnote{
%	Ainda que estas suposições sobre as situações admissíveis pareçam nada mais do que recomendações do senso comum, elas são controversas entre os filósofos da lógica.
%Em primeiro lugar, há lógicos que querem admitir situações em que as sentenças não são verdadeiras nem falsas, mas têm algum tipo de nível intermediário de verdade.
%De modo um pouco mais controverso, outros filósofos pensam que devemos permitir a possibilidade de que as sentenças sejam verdadeiras e falsas ao mesmo tempo. Existem sistemas de lógica, que não discutiremos neste livro, em que uma sentença pode tanto ser nem verdadeira nem falsa, quanto ser ambas, verdadeira e falsa.}


\section{Argumentos Corretos}
Argumentos são usados o tempo todo no discurso cotidiano e científico.
%Usamos um argumento quando queremos dar uma justificativa para a sua conclusão.
Muitas vezes não há como mostrar diretamente que uma sentença é verdadeira.
Aí, para comprovar sua verdade, usamos um argumento.
Conforme já dissemos várias vezes, quando um argumento é válido, suas premissas são uma garantia para a sua conclusão.
Mas um argumento válido só comprova que sua conclusão é verdadeira se todas as suas premissas forem verdadeiras também.
Conforme vimos na página \pageref{premissafalsa}, a conclusão de um argumento válido pode sim ser falsa quando pelo menos uma de suas premissas for falsa.
Ou seja, é perfeitamente possível que um argumento válido tenha conclusão falsa.

Considere o seguinte exemplo:
	\begin{earg}
		\item[] As laranjas ou são frutas ou são instrumentos musicais.
		\item[] As laranjas não são frutas.
		\item[\therefore] As laranjas são instrumentos musicais.
	\end{earg}
A conclusão desse argumento é ridícula. No entanto, ela se segue das premissas.
Se as duas premissas fossem verdadeiras, a conclusão também seria.
Por isso, apesar de ter uma conclusão falsa, esse argumento é válido.
Por outro lado, ter premissas e conclusão todas verdadeiras não garante que um argumento será válido.
Considere este exemplo:
	\begin{earg}
		\item[] Todo potiguar é brasileiro.
		\item[] Oscar Schmidt é brasileiro.
		\item[\therefore] Oscar Schmidt é potiguar.
	\end{earg}
As premissas e a conclusão desse argumento são todas verdadeiras, mas o argumento é inválido.
Se Oscar Schmidt tivesse nascido na Paraíba, as duas premissas continuariam sendo verdadeiras, mas a conclusão não seria.
Então há uma situação em que as premissas desse argumento são verdadeiras, mas a conclusão não é.
Portanto, o argumento é inválido.

Uma boa estratégia para não nos confundirmos é lembrar que a validade de um argumento não garante a verdade de sua conclusão, ela garante apenas que a informação da conclusão está contida na informação das premissas, independentemente de se elas são verdadeiras ou falsas no mundo real.
A situação que realmente ocorre, o mundo real, não tem nenhum papel especial na determinação de se um argumento é válido ou não.\footnote{
	O único caso em que o mundo real parece interferir na avaliação lógica de um argumento ocorre quando as premissas são de fato verdadeiras e a conclusão de fato é falsa.
	Neste caso o contraexemplo será um fato (real) e não uma possibilidade imaginada.
	Vivemos no contraexemplo, e o argumento é inválido.
	Mas a realidade do contraexemplo não o torna especial.
	De um ponto de vista lógico, um contraexemplo imaginário é tão forte quanto um contraexemplo real.}

A moral da história aqui é que se você quiser usar um argumento para comprovar que uma sentença é verdadeira, você tem de construir um argumento válido e com premissas verdadeiras no qual esta sentença seja a conclusão.
Um argumento com estas características, válido e com todas as suas premissas verdadeiras, é chamado de argumento \definepl{soundness}.

Por outro lado, se você quiser refutar um argumento, ou seja, se você quiser mostrar que um dado argumento não é uma boa justificativa para a verdade de sua conclusão, você pode fazer isso de dois modos diferentes:
ou você mostra que alguma das premissas não é verdadeira, ou você propõe um contraexemplo para o argumento e mostra que ele é inválido.

Dada uma afirmação qualquer, é sempre mais trabalhoso comprovar que ela é verdadeira do que duvidar de sua verdade.
Para \textit{comprovar} a verdade temos \textit{duas tarefas}, e para \textit{duvidar} da verdade, temos \textit{duas opções}.

\factoidbox{
	\begin{description}
		\item[Usar um argumento -- {\small (comprovar a verdade da conclusão)}] Se você quer usar um argumento para \textit{comprovar a verdade} de uma afirmação, você precisa que esta afirmação seja a conclusão de um argumento \textit{correto}. Ou seja, você tem \textbf{duas tarefas}:
		\begin{enumerate}
			\item  você \textit{tem de} mostrar que o argumento é válido, \textbf{e}
			\item você \textit{tem de} mostrar que todas as premissas são verdadeiras.
		\end{enumerate}
	\end{description}}
\factoidbox{
		\begin{description}
		\item[Refutar um argumento -- {\small (rejeitar a justificativa da conclusão)}] Se você quer refutar um argumento, ou seja, mostrar que ele não é uma boa justificativa para a verdade da conclusão, você tem \textbf{duas opções}:
		\begin{enumerate}
			\item você \textit{pode} mostrar que uma (ou mais) premissas são falsas, \textbf{ou}
			\item  você \textit{pode} mostrar que o argumento é inválido.
		\end{enumerate}
			\end{description}}
\factoidbox{
		\begin{description}
		\item[Falácia da falácia --] Cuidado!
			A refutação de um argumento não prova que sua conclusão é falsa.
			Ela prova apenas que o argumento não funciona como uma comprovação da verdade da conclusão.
			Para comprovar que a conclusão é falsa, é preciso propor um novo argumento que seja correto (válido e com premissas verdadeiras) e cuja conclusão seja a negação da conclusão do argumento refutado.
			O erro de achar que a refutação de um argumento prova que sua conclusão é falsa é conhecido como \textit{falácia da falácia}: é uma falácia (um erro de raciocínio) achar que o que uma pessoa diz é falso apenas porque a pessoa não tem uma boa justificativa para a sua afirmação.
		\end{description}}



\section{Argumentos Indutivos}
Dissemos antes que a classificação que a lógica faz dos argumentos é extremamente forte. 
Um argumento que a lógica reconhece como válido continuaria sendo válido mesmo se as leis científicas fossem totalmente diferentes do que elas são, e mesmo que as palavras tivessem significados totalmente diferentes dos que elas têm.
Essa força da avaliação lógica dos argumentos tem um lado positivo e um negativo.
O lado positivo é que a lógica é extremamente confiável.
Se um argumento é válido, a garantia que suas premissas dão para sua conclusão é infalível.
Você pode confiar sua vida num argumento válido, sem medo de errar.
O lado negativo é que ao ter padrões tão elevados, a lógica deixa de reconhecer como válidos muitos raciocínios bastante confiáveis, mas que, no entanto, não são infalíveis.
Considere o seguinte argumento:
	\begin{earg}
		\item[] Até hoje jamais nevou em Natal.
	\item[\therefore] Não nevará em Natal no próximo inverno.
\end{earg}
Esse argumento faz uma generalização baseada na observação sobre muitas situações (passadas) e conclui sobre uma situação (futura).
Tais argumentos são chamados argumentos \definepl{indutivo}.
Apesar de bastante razoável, esse argumento é inválido.
Mesmo que jamais tenha nevado em Natal até agora, permanece \emph{possível} que 
uma onda súbita de frio e umidade chegue a Natal no próximo inverno e neve na cidade.
Essa situação configura um cenário hipotético bastante implausível, mas ainda assim possível, em que a premissa do argumento é verdadeira, mas sua conclusão não é, o que caracteriza o argumento como inválido.

A questão importante aqui é que argumentos indutivos---mesmo os bons argumentos indutivos---são inválidos.
Eles não expressam raciocínios  \emph{infalíveis}.
Por mais improvável que seja, é \emph{possível} que sua conclusão seja falsa, mesmo quando todas as suas premissas são verdadeiras.

Os argumentos indutivos são um grupo de argumentos inválidos que nos interessam, porque mesmo não dando garantias infalíveis para a conclusão, um bom argumento indutivo nos dá um excelente motivo para acreditarmos na sua conclusão.
Apesar do exemplo acima ser um argumento inválido, eu confio tanto na sua conclusão de que não vai nevar em Natal no próximo inverno, que estou disposto a apostar dinheiro nela. Quer apostar comigo? Eu digo que não vai nevar!

Apesar de sua importância e utilidade, não abordaremos os argumentos indutivos neste livro.
Nosso interesse aqui é nos argumentos infalíveis.

Antes de finalizar o capítulo, aqui vão três alertas vocabulares:
\begin{enumerate}
	\item Você pode dizer que a conclusão \textit{se segue} das premissas, ou que é uma \textit{consequência} das premissas. Mas não diga que as premissas  \textit{inferem} a conclusão.
	Quem infere a conclusão a partir das premissas somos nós.
	 Inferência é um raciocínio, é algo que nós fazemos, não algo que as premissas fazem.
	\item Não diga que um argumento é verdadeiro ou é falso.
		Quem pode ser verdadeiras ou falsas são as sentenças.
		Os argumentos são válidos ou inválidos.
	\item Você talvez já tenha ouvido falar na distinção entre `dedução' e `indução'.
	Indução é o tipo de inferência (raciocínio) ligado aos argumentos indutivos, que não estudaremos neste livro. E dedução é o que estamos estudando aqui.
	Os conceitos de validade e consequência que começamos a estudar são também chamados de `validade dedutiva' e `consequência dedutiva'.
	A própria lógica que estamos estudando aqui é também chamada de `lógica dedutiva'.
\end{enumerate}


\practiceproblems
\problempart
Quais argumentos a seguir são válidos? Quais são inválidos?

\begin{enumerate}
\item
	\begin{earg}
		\item[] Sócrates é um homem.
		\item[] Todos os homens são repolhos.
		\item[\therefore] Sócrates é um repolho.
	\end{earg}

\item
	\begin{earg}
		\item[]  Lula nasceu em Porto Alegre ou foi presidente do Brasil.
		\item[] Lula nunca foi presidente do Brasil.
		\item[\therefore] Lula nasceu em Porto Alegre.
	\end{earg}

\item
	\begin{earg}
		\item[] Se eu acordar tarde eu me atrasarei.
		\item[] Eu não acordei tarde.
		\item[\therefore] Eu não me atrasei.
	\end{earg}

\item
	\begin{earg}
		\item[] Lula é gaúcho ou mato-grossense.
		\item[] Lula não é mato-grossense.
		\item[\therefore] Lula é gaúcho.
	\end{earg}

\item
	\begin{earg}
		\item[] Se o mundo acabar hoje, não precisarei acordar cedo amanhã.
		\item[] Precisarei acordar cedo amanhã.
		\item[\therefore] O mundo não vai acabar hoje.
	\end{earg}

\item
	\begin{earg}
		\item[] Lula tem hoje 74 anos.
		\item[] Lula tem hoje 39 anos.
		\item[\therefore] Lula tem hoje 50 anos.
	\end{earg}
\end{enumerate}

\problempart
\label{pr.EnglishCombinations}
Será que pode...
	\begin{earg}
		\item Um argumento válido com alguma premissa falsa e alguma verdadeira?
		\item Um argumento válido com todas as premissas falsas e a conclusão verdadeira?
		\item Um argumento válido com todas as premissas e também a conclusão falsa?
		\item Um argumento inválido com todas as premissas e também a conclusão verdadeiras?
		\item Um argumento válido com as premissas verdadeiras e a conclusão falsa?
		\item Um argumento inválido se tornar válido devido a adição de uma premissa extra?
		\item Um argumento válido se tornar inválido devido a adição de uma premissa extra?
	\end{earg}
Em cada caso, se pode, dê um exemplo, e se não pode, explique por que não.


\chapter{Outras noções lógicas}\label{s:BasicNotions}

No Capítulo \ref{s:Valid} apresentamos os conceitos mais importantes da lógica:
as noções correlatas de \textit{consequência} e  \textit{validade}.
Para defini-las introduzimos os conceitos de \textit{situação} e \textit{contraexemplo}. 
Não custa repetir:
um argumento é válido, bom do ponto de vista lógico, se a conclusão é consequência das premissas.
Isso ocorre  quando ele não tem contraexemplo, ou seja, quando não existe situação em que todas as suas premissas são verdadeiras e sua conclusão é falsa.
Nesse caso, a informação da conclusão já está contida na informação das premissas e por isso dizemos que as premissas \textit{garantem} (ou \textit{justificam}) a conclusão.

Exsitem, no entanto, outras definições importantes também derivadas da noção de situação que merecem nossa atenção.
Apresentaremos neste capítulo uma primeira versão delas, ainda em um contexto informal. Mais adiante, nas próximas partes do livro, voltaremos a estas definições e apresentaremos as suas versões formalizadas.
Então, como estamos em um contexto informal, no restante deste capítulo consideraremos como situações aceitáveis apenas aqueles cenários que respeitem nossos conceitos e a relação entre eles, com os quais definimos a validade conceitual (p. \pageref{nomoval}).

%O que dissemos até agora sobre os diferentes tipos de validade (validade nomológica, conceitual ou formal) pode ser estendido para as noções que apresentaremos neste capítulo. Ou seja, sempre que usarmos uma ideia diferente do que conta como uma ``situação'', teremos noções diferentes.
%E, enquanto lógicos, nós eventualmente consideraremos, mais adiante, uma definição de situação mais permissiva do que esta, da validade conceitual, apenas provisoriamente assumida.

\section{Compatibilidade}
Considere estas duas sentenças:
	\begin{ebullet}
		\item[B1.] O único irmão de Joana é mais baixo do que ela.
		\item[B2.] O único irmão de Joana é mais alto do que ela.
	\end{ebullet}	
A lógica não consegue, sozinha, nos dizer qual dessas frases é verdadeira.
No entanto, é evidente que \emph{se} a primeira sentença (B1) for verdadeira, \emph{então} a segunda (B2) deve ser falsa.
Da mesma forma, se a segunda (B2) for verdadeira, então a primeira (B1) deve ser falsa.
Não há cenário possível em que ambas as sentenças sejam verdadeiras juntas.
Essas sentenças são incompatíveis entre si, não podem ser ambas verdadeiras ao mesmo tempo.
Isso motiva a seguinte definição:

	\factoidbox{
		As sentenças de um grupo são \definepl{compatibility} se e somente se houver uma situação em que todas elas sejam verdadeiras.
	}
B1 e B2 são, então, \emph{incompatíveis}, ao passo que, as duas sentenças abaixo são compatíveis:\footnote{
	Sentenças compatíveis são também chamadas de \textit{conjuntamente possíveis} e as incompatíveis de \textit{conjuntamente impossíveis}.}
	\begin{ebullet}
		\item[B3.] O único irmão de Joana é mais baixo do que ela.
		\item[B4.] O único irmão de Joana é mais velho do que ela.
	\end{ebullet}

Podemos nos perguntar sobre a compatibilidade de um número qualquer de sentenças.
Por exemplo, considere as seguintes quatro sentenças:
	\begin{ebullet}	
		\item[G1.] \label{MartianGiraffes} Há pelo menos quatro girafas no zoológico de Natal.
		\item[G2.] Há exatamente sete gorilas no zoológico de Natal.
		\item[G3.] Não há mais do que dois marcianos no zoológico de Natal.
		\item[G4.] Cada girafa no zoológico de Natal é um marciano.
	\end{ebullet}
G1 e G4 juntas implicam que há pelo menos quatro girafas marcianas no zoológico de Natal.
E isso entra em conflito com o G3, que afirma não haver mais de dois marcianos no zoológico de Natal.
Portanto, as sentenças G1, G2, G3, G4 são incompatíveis.
Não há situação na qual todas elas sejam verdadeiras. 
(Observe que as sentenças G1, G3 e G4, sem G2, já são incompatíveis.
E se elas são incompatíveis, adicionar uma sentença extra ao grupo, como G2, não resolverá esta incompatibilidade!)

\section[Verdades e falsidades necessárias e contingência]{Verdades necessárias, falsidades necessárias e contingências}

Quando avaliamos se um argumento é válido ou não, nossa preocupação é com o que seria verdadeiro \emph{se} as premissas fossem verdadeiras.
Mas além das sentenças que podem ser verdadeiras e podem não ser, existem algumas sentenças que simplesmente têm que ser verdadeiras, que não é possível que não sejam verdadeiras.
E existem outras que simplesmente têm que ser falsas, que não é possível que sejam verdadeiras.
Considere as seguintes sentenças:
	\begin{earg}
		\item[\ex{Acontingent}] Está chovendo.
		\item[\ex{Atautology}] Ou está chovendo, ou não está.
		\item[\ex{Acontradiction}] Está e não está chovendo.
	\end{earg}
Para saber se a sentença \ref{Acontingent} é verdadeira, você precisa olhar pela janela. Ela pode ser verdadeira, mas também pode ser falsa.
Uma sentença como \ref{Acontingent} que é capaz de ser verdadeira e capaz de ser falsa (em diferentes circunstâncias, é claro) é chamada \define{contingent}.

A sentença \ref{Atautology} é diferente.
Você não precisa olhar pela janela para saber que ela é verdadeira. Independentemente de como está o tempo, ou está chovendo ou não está.
Uma sentença como \ref{Atautology} que é incapaz de ser falsa, ou seja, que é verdadeira em qualquer situação, é chamada de \define{necessary truth}.

Da mesma forma, você não precisa verificar o clima para saber se a sentença \ref{Acontradiction} é verdadeira.
Ela não pode ser verdadeira, tem de ser falsa.
Pode estar chovendo aqui e não estar chovendo em outro lugar; pode estar chovendo agora, e parar de chover antes mesmo de você terminar de ler esta sentença; mas é impossível que esteja e não esteja chovendo no mesmo lugar e ao mesmo tempo.
Então, independentemente de como seja o mundo, a sentença \ref{Acontradiction}, que afirma que está e não está chovendo, é falsa.
Uma sentença como \ref{Acontradiction}, que é incapaz de ser verdadeira, ou seja, que é falsa em qualquer situação, é uma \define{necessary falsehood}.

Algo, porém, pode ser verdadeiro \emph{sempre} e, mesmo assim, ainda ser contingente. Por exemplo, se é verdade que nenhum ser humano jamais viveu 150 anos ou mais, então a sentença `Nenhum ser humano morreu com 150 anos ou mais' sempre foi verdadeira.
Apesar disso, esta ainda é uma sentença contingente, porque podemos conceber um cenário hipotético, uma situação, na qual os seres humanos vivem mais do que 150 anos.
Então a sentença é e sempre foi verdadeira, mas poderia ser falsa e é, por isso, uma sentença contingente.

\subsection{Equivalência Necessária}

Podemos também perguntar sobre as relações lógicas \emph{entre} duas sentenças.
Por exemplo:
\begin{earg}
\item[] Isabel foi trabalhar depois de lavar a louça.
\item[] Isabel lavou a louça antes de ir trabalhar.
\end{earg}
Essas duas sentenças são contingentes, pois Isabel poderia não ter lavado a louça, ou não ter ido trabalhar.
No entanto, se uma delas for verdadeira, a outra também será; se uma for falsa, a outra também será.
Quando duas sentenças têm sempre o mesmo valor de verdade em todas as situações, dizemos que elas são \definepl{necessary equivalence}.

\section*{Sumário das noções lógicas}

\begin{itemize}
\item Um argumento é \textit{válido} se não houver nenhuma situação em que as premissas sejam todas verdadeiras e a conclusão seja falsa; é \textit{inválido} caso contrário.

\item Uma \textit{verdade necessária} é uma sentença verdadeira em todas as situações.

\item Uma \textit{falsidade necessária} é uma sentença que é falsa em todas as situações.

\item Uma \textit{sentenca contingente} não é nem uma verdade necessária nem uma falsidade necessária; é verdadeira em algumas situações e falsa em outras.

\item Duas sentenças são \textit{necessariamente equivalentes} se, em qualquer situação, têm o mesmo valor de verdade; ou são ambas verdadeiras ou ambas falsas.

\item As sentenças de uma coleção são \textit{compatíveis} se houver uma situação em que todas sejam verdadeiras; e são \textit{incompatíveis} caso contrário.
\end{itemize}


\practiceproblems
\problempart
\label{pr.EnglishTautology2}
Para cada uma das sentenças seguintes, decida se ela é uma verdade necessária, uma falsidade necessária, ou se é contingente.
\begin{earg}
\item Penélope atravessou a estrada.
\item Alguém uma vez atravessou a estrada.
\item Ninguém jamais atravessou a estrada.
\item Se Penélope atravessou a estrada, então alguém atravessou.
\item Embora Penélope tenha atravessado a estrada, ninguém jamais atravessou a estrada.
\item Se alguém alguma vez atravessou a estrada, foi Penélope.
\end{earg}

\problempart
Para cada uma das sentenças seguintes, decida se ela é uma verdade necessária, uma falsidade necessária, ou se é contingente.
\begin{earg}
\item Os elefantes dissolvem na água.
\item A madeira é uma substância leve e durável, útil para construir coisas.
\item Se a madeira fosse um bom material de construção, seria útil para construir coisas.
\item Eu moro em um prédio de três andares que é de dois andares.
\item Se os calangos fossem mamíferos, eles amamentariam seus filhotes.
\end{earg}

\problempart Quais pares abaixo possuem sentenças necessariamente equivalentes?

\begin{earg}
\item Os elefantes dissolvem na água.	\\
	Se você colocar um elefante na água, ele irá se desmanchar.
\item Todos os mamíferos dissolvem na água. \\		
	Se você colocar um elefante na água, ele irá se desmanchar. 
\item Lula foi o 4º presidente depois da ditadura militar de 64--85. \\
	 Dilma foi a 5ª presidenta depois da ditadura militar de 64--85. 
\item Dilma foi a 5ª presidenta depois da ditadura militar de 64--85. \\
	  Dilma foi a presidenta imediatamente após o 4º presidente depois da ditadura militar de 64--85.
\item Os elefantes dissolvem na água. 	\\	
	Todos os mamíferos dissolvem na água. 
\end{earg}

\problempart Quais pares abaixo possuem sentenças necessariamente equivalentes?

\begin{earg}
\item Luiz Gonzaga tocava sanfona. \\
	Jackson do Pandeiro tocava pandeiro.

\item Luiz Gonzaga tocou junto com Jackson do Pandeiro. \\
	Jackson do Pandeiro tocou junto com Luiz Gonzaga.

\item Todas  as pianistas profissionais têm mãos grandes. \\
	A pianista Nina Simone tinha mãos grandes.

\item Nina Simone tinha a saúde mental abalada. \\
	Todos os pianistas têm a saúde mental abalada.

\item Roberto Carlos é profundamente religioso. \\
	Roberto Carlos concebe a música como uma expressão de sua espiritualidade.
\end{earg}


\noindent \problempart Considere as seguintes sentenças: 
\begin{enumerate}%[label=(\alph*)]
\item[G1.] \label{itm:at_least_four} Há pelo menos quatro girafas no zoológico de Natal.
\item[G2.] \label{itm:exactly_seven} Há exatamente sete gorilas no zoológico de Natal.
\item[G3.] \label{itm:not_more_than_two} Não há mais do que dois marcianos no zoológico de Natal.
\item[G4.] \label{itm:martians} Cada girafa do zoológico de Natal é um marciano.
\end{enumerate}

Agora considere cada uma das seguintes coleções de sentenças. Quais têm sentenças compatíveis? E em quais as sentenças são incompatíveis?
\begin{earg}
\item Sentenças G2, G3 e G4
\item Sentenças G1, G3 e G4
\item Sentenças G1, G2 e G4
\item Sentenças G1, G2 e G3
\end{earg}

\problempart Considere as seguintes sentenças.
\begin{enumerate}%[label=(\alph*)]
\item[M1.] \label{itm:allmortal} Todas as pessoas são mortais.
\item[M2.] \label{itm:socperson} Sócrates é uma pessoa.
\item[M3.] \label{itm:socnotdie} Sócrates nunca morrerá.
\item[M4.] \label{itm:socmortal} Sócrates é mortal.
\end{enumerate}
Em quais dos 5 grupos abaixo as sentenças são compatíveis? E em quais elas são incompatívieis?
\begin{earg}
\item Sentenças M1, M2, e M3
\item Sentenças M2, M3, e M4
\item Sentenças M2 e M3
\item Sentenças M1 e M4
\item Sentenças M1, M2, M3 e M4
\end{earg}

\problempart
\label{pr.EnglishCombinations2}
Para cada item abaixo, decida se ele é possível ou impossível.
Se for possível apresente um exemplo.
Se for impossível, explique por quê.
\begin{earg}
\item Um argumento válido que possui uma premissa falsa e uma premissa verdadeira.

\item Um argumento válido que tem a conclusão falsa.

\item Um argumento válido, cuja conclusão é uma falsidade necessária.

\item Um argumento inválido, cuja conclusão é uma verdade necessária.

\item Uma verdade necessária que é contingente.

\item Duas sentenças necessariamente equivalentes, ambas verdades necessárias.

\item Duas sentenças necessariamente equivalentes, uma das quais é uma verdade necessária e uma das quais é contingente.

\item Duas sentenças necessariamente equivalentes que são incompatíveis.

\item Um grupo de sentenças compatíveis que contém uma falsidade necessária.

\item Um grupo de sentenças incompatíveis que contém uma verdade necessária.
\end{earg}


\problempart
Para cada item abaixo, decida se ele é possível ou impossível.
Se for possível apresente um exemplo.
Se for impossível, explique por quê.
\begin{earg}
\item Um argumento válido, cujas premissas são todas verdades necessárias e cuja conclusão é contingente.
\item Um argumento válido com premissas verdadeiras e conclusão falsa.
\item Um grupo de sentenças compatíveis que contém duas sentenças que não são necessariamente equivalentes.
\item Uma coleção de sentenças compatíveis, todas elas contingentes.
\item Uma verdade necessária falsa.
\item Um argumento válido com premissas falsas.
\item Um par de sentenças necessariamente equivalentes que não são compatíveis.
\item Uma verdade necessária que também é uma falsidade necessária.
\item Um grupo de sentenças compatíveis que também são falsidades necessárias.
\end{earg}

