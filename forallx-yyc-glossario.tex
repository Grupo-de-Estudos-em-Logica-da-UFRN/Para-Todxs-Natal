% Glossário - Versão Para Todxs - Natal 

% Criei o comando \definex{} em  forallsyyc.sty - funciona com acento e tem link.

\makeglossaries

\newacronym{lvf}{LVF}{-- lógica verofuncional}

\newacronym{lpo}{LPO}{-- lógica de primeira ordem}

\newglossaryentry{premise indicator word}
{
name=expressão indicativa de premissa,
plural=expressões indicativas de premissa,
description={-- expressão ou palavra como ``porque'' usada para indicar que o que se segue é uma premissa de um argumento}
}

\newglossaryentry{conclusion indicator word}
{
name=expressão indicativa de conclusão,
plural=expressões indicativas de conclusão,
description={-- expressão ou palavra como ``portanto'' usada para indicar que o que se segue é a conclusão de um argumento}
}

\newglossaryentry{argumento}
{
name=argumento,
plural=argumentos,
description={-- um grupo de sentenças que corresponde à expressão linguística de um raciocínio. Uma delas é a conclusão a que se chega e todas as outras são as premissas das quais se parte}
}

\newglossaryentry{premise}
{
name=premissa,
plural=premissas,
description={-- qualquer sentença de um argumento que não é a conclusão}
}

\newglossaryentry{conclusion}
{
name=conclusão,
plural=conclusões,
description={-- a sentença principal de um argumento, aquela inferida a partir das premissas}
}

%\newglossaryentry{valid}
%{
%name=valid,
%description={A property of arguments where there conclusion is a consequence of the premises}
%}

%\newglossaryentry{invalid}
%{
%name=invalid,
%description={A property of arguments that holds when the conclusion is not a conseqeucne of the premises; the opposite of %\gls{valid}} 
%}

\newglossaryentry{soundness} %correcao
{
name=correção,
plural=correto,
description={-- (1) propriedade possuída pelos argumentos que são válidos e têm todas as premissas verdadeiras; (2) propriedade de sistemas lógicos para os quais $\proves$ implica $\entails$}
}

\newglossaryentry{compatibility} %possibility
{
name=compatibilidade,
plural=compatíveis,
text={conjuntamente possíveis}, %jointly possible
description={-- propriedade de grupos de sentenças que não se contradizem mutuamente e podem ser todas verdadeiras em uma mesma situação}
}

\newglossaryentry{contingent sentence}
{
name=sentença contingente,
plural=sentenças contingentes,
description={-- uma sentença que não é nem uma verdade necessária, nem uma falsidade necessária, mas que é verdadeira em algumas situações e falsa em outras.}
}


\newglossaryentry{necessary truth}
{
name={verdade necessária},
plural={verdades necessárias},
description={-- uma sentença que é verdadeira em qualquer situação, que é impossível de ser falsa}
}

\newglossaryentry{necessary falsehood}
{
name={falsidade necessária},
plural={falsidades necessárias},
description={-- uma sentença que é falsa em qualquer situação, que é impossível de ser verdadeira}
}

\newglossaryentry{necessary equivalence}
{
name={equivalência necessária},
plural={equivalências necessárias},
%text={necessariamente equivalente},
description={-- propriedade de um grupo de sentenças que em qualquer situação têm sempre o mesmo valor de verdade, ou são todas verdadeiras, ou todas falsas}
}

\newglossaryentry{sentence letter}
{
name=letra sentencial,
plural=letras sentenciais,
description={-- (1) letra maiúscula possivelmente com índice numérico usada para representar na LVF uma sentença simples do português; (2) um dos elementos básicos constituintes da linguagem da LVF}
}
\newglossaryentry{atomic sentence}
{
name=sentença atômica,
plural=sentenças atômicas,
description={-- expressão usada para representar uma sentença simples; na LVF corresponde a uma letra sentencial; na LPO corresponde a um símbolo de predicado seguido de nomes}
}

\newglossaryentry{symbolization key}
{
name=chave de simbolização,
plural=chaves de simbolização,
description={-- (1) na LVF é uma lista que relaciona letras sentenciais com as sentenças simples do português que elas representam (simbolizam); (2) na LPO, além desta lista, contém também a definição do domínio do discurso, a relação dos indivíduos do domínio representados por cada s e a indicação da intensão (ou significado) de cada símbolo de predicado}
}

\newglossaryentry{connective}
{
name=conectivo,
plural=conectivos,
description={-- (1) um operador lógico da LVF usado para combinar letras sentenciais em sentenças maiores; (2) um dos elementos básicos constituintes da linguagem da LVF}
}

\newglossaryentry{negation}
{
name=negação,
description={-- (1) nome do conectivo `\enot', usado para simbolizar expressões do português que tenham o mesmo significado da palavra ``não''; (2) nome das sentenças cujo conectivo principal é `\enot'}
}

\newglossaryentry{conjunction}
{
name=conjunção,
description={-- (1) nome do conectivo `\eand', usado para simbolizar expressões do português que tenham o mesmo significado da palavra ``e''; (2) nome das sentenças cujo conectivo principal é `\eand'}
}

\newglossaryentry{conjunct}
{
name=conjunto,
plural=conjuntos,
description={-- qualquer uma das duas sentenças concatenadas por `\eand' em uma conjunção}
}

\newglossaryentry{disjunction}
{
name=disjunção,
description={-- (1) nome do conectivo `\eor', usado para simbolizar expressões do português que tenham o mesmo significado da palavra ``ou'' em seu uso inclusivo; (2) nome das sentenças cujo conectivo principal é `\eor'}
}

\newglossaryentry{disjunct}
{
name=disjunto,
plural=disjuntos,
description={-- qualquer uma das duas sentenças conectadas por `\eor' em uma disjunção}
}

\newglossaryentry{conditional}
{
name=condicional,
plural=condicionais,
description={-- (1) nome do conectivo `\eif', usado para simbolizar expressões do português que significam o mesmo que ``se\ldots{}então\ldots''; (2) nome das sentenças cujo conectivo principal é `\eif'}
}

\newglossaryentry{antecedent}
{
name=antecedente,
plural=antecedentes,
description={-- a sentença à esquerda do símbolo `\eif' em um condicional}
}

\newglossaryentry{consequent}
{
name=consequente,
plural=consequentes,
description={-- a sentença a direita do símbolo `\eif' em um condicional}
}

\newglossaryentry{biconditional}
{
name=bicondicional,
plural=bicondicionais,
description={-- (1) nome do conectivo `\eiff', usado para simbolizar expressões do português que significam o mesmo que ``se e somente se''; (2) nome das sentenças cujo conectivo principal é `\eiff'}
}

\newglossaryentry{sentence of TFL}
{
name=sentença (da LVF),
plural=sentenças (da LVF),
description={-- uma sequência de símbolos da LVF construída de acordo com as regras indutivas apresentadas na p.~\pageref{TFLsentences}}
}

\newglossaryentry{main logical operator}
{
name=conectivo principal,
description={-- o conectivo mais externo com relação aos parênteses em uma sentença; o último conectivo adicionado quando se constrói uma sentença de acordo com a definição indutiva}
}

\newglossaryentry{object language}
{
name=linguagem objeto,
description={-- é a linguagem que está sendo estudada e construída. Neste livro, as duas linguagens objetos abordadas são a LVF e a LPO}
}

\newglossaryentry{metalanguage}
{
name=metalinguagem,
description={-- é a linguagem que os lógicos usam para falar sobre a linguagem objeto. Neste livro, a metalinguagem é o português aumentado por certos símbolos, como as metavariáveis e certos termos técnicos, como ``valido''}
}

\newglossaryentry{metavariables}
{
name=metavariável,
plural=metavariáveis,
description={-- uma variável da metalinguagem que pode representar qualquer sentença da linguagem objeto}
}

\newglossaryentry{truth value}
{
name = valor de verdade,
plural= valores de verdade,
description = {-- um dos dois valores lógicos que as sentenças podem ter: o Verdadeiro e o Falso}
}

\newglossaryentry{truth-functional connective}
{
name=verofuncional,
description={-- qualquer conectivo em que o valor de verdade da sentença em que ele é o conectivo principal é determinado univocamente pelos valores de verdade das sentenças componentes}
}

\newglossaryentry{valuation}
{
name=valoração,
description={-- (1) uma atribuição de valores de verdade a um grupo específico de letras sentenciais; (2) qualquer linha de uma tabela de verdade}
}

\newglossaryentry{complete truth table}
{
name=tabela de verdade completa,
description={-- uma tabela que apresenta todos os valores de verdade possíveis para uma (ou um grupo de) sentença(s) da LVF}
}

\newglossaryentry{tautology}
{
name=tautologia,
plural=tautologias,
description={-- uma sentença que é verdadeira em todas as valorações; ou seja, uma sentença cuja tabela de verdade tem V em todas as linhas na coluna abaixo do conectivo principal}
}

\newglossaryentry{contradiction of TFL}
{
  name=contradição (na LVF),
  text = contradição,
description={-- uma sentença que é falsa em todas as valorações; ou seja, uma sentença cuja tabela de verdade tem F em todas as linhas na coluna abaixo do conectivo principal}
}

\newglossaryentry{equivalent}
{
  name=equivalência (na LVF),
  text = equivalente,
  plural=equivalentes,
description={-- propriedade de um grupo de sentenças que em uma tabela de verdade conjunta completa têm exatamente os mesmos valores de verdade sob a coluna do conectivo principal em cada linha da tabela. Se uma é V, todas as outras também são, e se uma é F, todas as outras também são; ou seja, a propriedade de um grupo de sentenças que em cada valoração têm valores de verdade idênticos umas aos das outras}
}

\newglossaryentry{satisfiability in TFL}
{
  name=compatibilidade (na LVF),
  text=conjuntamente satisfatórias (ou compatíveis),
description={-- (também designada de satisfatoriedade conjunta) propriedade de um grupo de sentenças da LVF para as quais em uma tabela de verdade conjunta há pelo menos uma linha em que todas as sentenças são verdadeiras; ou seja, a propriedade de um grupo de sentenças para o qual há pelo menos uma valoração na qual todas as sentenças são verdadeiras}
}

\newglossaryentry{valid in TFL}
{
name= validade de argumentos (na LVF),
text = válido,
description={-- propriedade de argumentos para os quais não há linha na tabela de verdade completa do argumento na qual todas as premissas sejam verdadeiras e a conclusão seja falsa; ou seja, a propriedade de argumentos para os quais não há valorações nas quais as premissas sejam todas verdadeiras e a conclusão seja falsa}
}

\newglossaryentry{name}{
name = nome,
plural=nomes,
description = {-- um símbolo da LPO usado para designar um objeto específico do domínio}
}

\newglossaryentry{predicate}{
name = predicado,
plural=predicados,
description = {-- um símbolo da LPO usado para simbolizar uma propriedade ou relação}
}

\newglossaryentry{universal quantifier}{
name = quantificador universal,
description = {-- nome do símbolo `$\forall$' da LPO usado para simbolizar uma afirmação geral: `$\forall x\, \atom{F}{x}$' é verdadeira se e somente se todos os indivíduos do domínio satisfazem o predicado $F$}
}

\newglossaryentry{variable}{
name = variável,
description = {-- (1) um símbolo da LPO usado logo após os quantificadores e também como marcador de posição em fórmulas atômicas; (2) letras minúsculas, possivelmente com índices numéricos, entre $s$ e~$z$}
}

% % daqui para baixo precisa apagar nos arquivos fonte

\newglossaryentry{existential quantifier}{
  name = quantificador existencial,
  description = {-- nome do símbolo `$\exists$' da LPO usado para simbolizar uma afirmação de existência: `$\exists x\, \atom{F}{x}$' é verdadeira se e somente se pelo menos um indivíduo do domínio satisfaz o predicado~$F$}
}

\newglossaryentry{domain}{
  name = domínio,
  description = {-- (1) a totalidade de indivíduos assumida por uma simbolização na LPO; (2) o universo do discurso; (3) a totalidade assumida como alcance das variáveis e quantificadores em uma dada interpretação}
}

\newglossaryentry{empty predicate}{
  name = {predicado vazio},
  description = {-- um predicado que não se aplica a nenhum indivíduo do domínio}
}

\newglossaryentry{term}{
  name = termo,
  plural=termos,
  description = {-- um nome ou uma variável}
}

\newglossaryentry{formula}{
  name = fórmula,
  plural=fórmulas,
  description = {-- uma expressão (sequência de símbolos) da LPO construída de acordo com as regras indutivas apresentadas na Seção~\S\ref{s:TermsFormulas}}
}

% retirar - já tem conectivo principal
%\newglossaryentry{main logical operator}{
%  name = operador lógico principal,
%  description = {The operator used last in the construction of a \gls{sentence of TFL} or a \gls{formula} of FOL}
%}

% incluir também âmbito
\newglossaryentry{scope}{
  name = escopo,
  description = {-- dado um conectivo, seu escopo (ou âmbito) é(são) a(s) subfórmula(s) da fórmula em que este conectivo é o conectivo principal}
}

\newglossaryentry{bound variable}{
  name = variável ligada,
  plural=variáveis ligadas,
  description = {-- uma variável $\meta{x}$ é ligada em uma fórmula da LPO se todas as suas ocorrências nesta fórmula forem ligadas}
}

\newglossaryentry{ocorrencia ligada}{
  name = ocorrência ligada,
  plural=ocorrências ligadas,
  description = {-- uma ocorrência de uma variável $\meta{x}$ é ligada se estiver dentro do escopo de um quantificador $\forall{\meta{x}}$ ou $\exists{\meta{x}}$}
}

\newglossaryentry{ocorrencia livre}{
  name = ocorrência livre,
  plural=ocorrências livres,
  description = {-- uma ocorrência de uma variável $\meta{x}$ é livre se \textbf{não} estiver dentro do escopo de um quantificador $\forall{\meta{x}}$ ou $\exists{\meta{x}}$}
}

\newglossaryentry{free variable}{
  name = variável livre,
  plural = variáveis livres,
  description = {-- uma variável $\meta{x}$ é livre em uma fórmula da LPO se ela tiver pelo menos uma ocorrência livre nesta fórmula}
}

\newglossaryentry{sentence of FOL}{
	name = sentença (da LPO),
	text = sentença da LPO,
	description = {-- uma fórmula da LPO que não tem variáveis livres. Uma setença também é chamada de \textit{fórmula fechada}}
}

\newglossaryentry{interpretation}{
  name = {interpretação},
  description = {-- a especificação de um domínio do discurso, dos indivíduos deste domínio referidos pelos nomes e dos objetos (sequências de objetos) que satisfazem os predicados (relações)}
}

\newglossaryentry{substitution instance}{
  name = instância de substituição,
  description = {-- o resultado de substituir todas as ocorrências livres de uma variável em uma fórmula por um nome}
}

\newglossaryentry{validity}
{
name=validade lógica (na LPO),
description={-- uma sentença da LPO que é verdadeira em todas as interpretações}
}

\newglossaryentry{contradiction of FOL}
{
  name=contradição (na LPO),
  text=contradição,
description={--  uma sentença da LPO que é falsa em todas as interpretações}
} 

\newglossaryentry{valid in FOL}
{
  name=validade de argumentos (na LPO),
  text = válido,
description={-- a propriedade dos argumentos da LPO para os quais não há nenhuma interpretação na qual todas as suas premissas são verdadeiras e a conclusão é falsa}
}

\newglossaryentry{equivalent in FOL}
{
  name=equivalência (na LPO),
  text = equivalente,
  plural=equivalentes,
description={-- a propriedade de um grupo de sentenças que, em qualquer interpretação, cada sentença do grupo tem sempre o mesmo valor de verdade que todas as demais}
}

\newglossaryentry{satisfiable in FOL}
{
  name=compatibilidade (na LPO),
  text=compatíveis ou conjuntamente satisfatórias,
description={-- (ou satisfatoriedade conjunta) a propriedade de um grupo de sentenças da LPO para as quais há pelo menos uma interpretação na qual todas as sentenças do grupo são verdadeiras}
}

\newglossaryentry{disjunctive normal form}
{
  name = forma normal disjuntiva (FND),
  text = forma normal disjuntiva,
  description = {-- está na forma normal disjuntiva uma sentença que é uma disjunção de conjunções de sentenças atômicas ou negações de sentenças atômicas}
}

\newglossaryentry{forma normal conjuntiva}
{
  name = forma normal conjuntiva (FNC),
  text = forma normal conjuntiva,
  description = {-- está na forma normal conjuntiva uma sentença que é uma conjunção de disjunções de sentenças atômicas ou de negações de sentenças atômicas}
}

\newglossaryentry{expressivamente adequado}{
  name = adequação expressiva,
  description = {-- a propriedade de um grupo de conectivos com os quais é possível construir todas as tabelas de verdade possíveis através de sentenças que contenham apenas conectivos deste grupo}
}
  
  \newglossaryentry{teorema}
  {
  name=teorema,
  description={-- uma sentença que é provada a partir de nenhuma premissa}
  }
  
  \newglossaryentry{dedutivamente equivalente}
  {
    name=dedutivamente equivalente,
    text = dedutivamente equivalente,
    plural=dedutivamente equivalentes,
  description={-- duas sentenças `$A$' e `$B$' são  dedutivamente equivalentes  se e somente se cada  uma puder ser provada a partir da outra}
  }
  
  \newglossaryentry{dedutivamente inconsistente}
  {    name={dedutivamente inconsistente}, 
       plural={dedutivamente inconsistentes},
    description={-- a propriedade de um grupo de sentenças a partir do qual se pode provar uma contradição}
  }
  
  \newglossaryentry{completude}
  {
  name=completude,
  description={-- a propriedade dos sistemas lógicos nos quais $\entails $ implica $\proves$}
  }
  
  \newglossaryentry{sentence}
  {
  name=sentença,
  plural=sentenças,
  description={-- frase ou expressão linguística completa, capaz de ser julgada como verdadeira ou falsa}
  }
  
  \newglossaryentry{valid}
  {
  name=argumento válido,
  plural=válidos,
  description={-- um argumento que não tem contraexemplo, no qual em qualquer situação na qual todas suas premissas são verdadeiras, sua conclusão também é}
  }
  
  \newglossaryentry{invalid}
  {
  name=inválido,
  plural=inválidos,
  description={-- um argumento que possui contraexemplo, para o qual existe alguma situação na qual suas premissas são verdadeiras e sua conclusão é falsa}
  }
  
  \newglossaryentry{counterexample}
  {
  name=contraexemplo,
  plural=contraexemplos,
  description={-- uma situação que demonstra a invalidade de um argumento, na qual todas as premissas são verdadeiras, mas sua conclusão é falsa}
  }
  
  \newglossaryentry{nomologicamente valido}
  {
  name=nomologicamente válido,
  plural=nomologicamente válidos,
  description={-- um argumento para o qual não há contraexemplos que não violem alguma lei da natureza}
  }
  
  \newglossaryentry{conceitualmente valido}
  {
  name=conceitualmente válido,
  plural=conceitualmente válidos,
  description={-- um argumento para o qual não há contraexemplos que não violem as conexões de nossos conceitos, ou seja, o significado das palavras}
  }
  
  \newglossaryentry{formalmente valido}
  {
  name=formalmente válido,
  plural=formalmente válidos,
  description={-- o mesmo que argumento válido; um argumento que não tem contraexemplo}
  }
  
  \newglossaryentry{indutivo}
  {
  name=indutivo,
  plural=indutivos,
  text=argumento indutivo,
  description={-- argumento que faz uma generalização baseada em observação sobre muitas situações (passadas) e conclui sobre todas as situações (futuras)}
  }
  
