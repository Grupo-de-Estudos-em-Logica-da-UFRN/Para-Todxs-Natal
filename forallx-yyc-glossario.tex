% Glossário - Versão Para Todxs - Natal 

% Criei o comando \definex{} em  forallsyyc.sty - funciona com acento e tem link.

\makeglossaries

%\newacronym{lvf}{LVF}{-- lógica verofuncional}

%\newacronym{lpo}{LPO}{-- lógica de primeira ordem}

\newglossaryentry{premise indicator word}
{
 name=expressão indicativa de premissa,
 plural=expressões indicativas de premissa,
 description={-- expressão ou palavra como ``porque'' usada para indicar que o que se segue é uma premissa de um argumento}
}

\newglossaryentry{conclusion indicator word}
{
 name=expressão indicativa de conclusão,
 plural=expressões indicativas de conclusão,
 description={-- expressão ou palavra como ``portanto'' usada para indicar que o que se segue é a conclusão de um argumento}
}

\newglossaryentry{argument}
{
 name=argumento,
 plural=argumentos,
 description={-- um grupo de sentenças que corresponde à expressão linguística de um raciocínio. Uma delas é a conclusão a que se chega e todas as outras são as premissas das quais se parte}
}

\newglossaryentry{premise}
{
 name=premissa,
 plural=premissas,
 description={-- (1) uma das sentenças de um argumento da qual se parte para inferir a conclusão; (2) qualquer sentença de um argumento que não é a conclusão.}
}

\newglossaryentry{possible world}
{
 name=mundo possível,
 plural=mundos possíveis,
 description={-- Um mundo possível é uma versão alternativa para o mundo real, onde um ou mais fatos são substituídos por versões alternativas que não ocorrem (ou ocorreram) na realidade, mas que poderiam ocorrer (ou ter ocorrido)}
}

\newglossaryentry{conclusion}
{
 name=conclusão,
 plural=conclusões,
 description={-- a sentença principal de um argumento, aquela inferida a partir das premissas}
}

\newglossaryentry{soundness} %correcao
{
 name=correção,
 plural=correto,
 description={-- (1) propriedade possuída pelos argumentos que são válidos e têm todas as premissas verdadeiras; (2) propriedade de sistemas lógicos para os quais $\proves$ implica $\entails$}
}

\newglossaryentry{compatibility} %possibility
{
 name=compatibilidade,
 plural=compatíveis,
 text={conjuntamente possíveis}, %jointly possible
 description={-- (ou \textit{possibilidade conjunta}) propriedade de grupos de sentenças que não se contradizem mutuamente e podem ser todas verdadeiras em uma mesma situação}
}

\newglossaryentry{contingent}
{
 name=contingência,
 plural=contingentes,
 text=contingente,
 description={-- (ou \textit{sentença contingente}) uma sentença que pode ser verdadeira em algumas situações e falsa em outras. Ou seja, uma sentença que poderia ter um valor de verdade diferente do que tem; cuja verdade ou falsidade não é necessária}
}

\newglossaryentry{exclusive or}
{
 name=ou exclusivo,
 plural=ous exclusivos,
 text=ou exclusivo,
 description={-- quando a palavra `ou' é usada de uma maneira que exclui a
possibilidade de que ambos os disjuntos sejam verdadeiros}
}

\newglossaryentry{inclusive or}
{
 name=ou inclusivo,
 plural=ous inclusivos,
 text=ou inclusivo,
 description={-- quando a palavra `ou' permite a possibilidade de ambos os disjuntos
serem verdadeiros}
}

\newglossaryentry{TFL expression}
{
 name=expressão da LVF,
 plural=expressões da LVF,
 text=expressão da LVF,
 description={-- qualquer sequência de símbolos da LVF}
}

\newglossaryentry{FOL expression}
{
 name=expressão da LPO,
 plural=expressões da LPO,
 text=expressão da LPO,
 description={-- qualquer sequência de símbolos da LPO}
}

\newglossaryentry{necessary truth}
{
 name={verdade necessária},
 plural={verdades necessárias},
 description={-- uma sentença que é verdadeira em qualquer situação; que é impossível de ser falsa}
}

\newglossaryentry{natural deduction}
{
 name={dedução natural},
 plural={deduções naturais},
 description={-- (1) um sistema dedutivo utilizado para construir provas formais a partir de um conjunto de regras de inferências. (2) um sistema de regras de inferência que propicia um método de verificação da validade de argumentos.}
}

\newglossaryentry{introduction rule}
{
 name={regra de introdução},
 plural={regras de introdução},
 description={-- são regras que nos permitem provar uma sentença que tenha esse conectivo como operador lógico principal}
}

\newglossaryentry{elimination rule}
{
 name={regra de eliminação},
 plural={regras de eliminação},
 description={-- são regras que nos permitem provar algo a partir de uma
sentença que tenha esse conectivo como operador lógico principal}
}

\newglossaryentry{syntactic contradiction in TFL}
{
 name={contradição sintática na LVF},
 plural={contradições sintáticas na LVF},
 description={-- uma sentença cuja negação pode ser provada sem quaisquer premissas}
}

\newglossaryentry{syntactic contingency in TFL}
{
 name={contingência sintática na LVF},
 plural={contingências sintáticas na LVF},
 text={sintaticamente contingente},
 description={-- uma sentença é sintaticamente contingente na LVF se ela não for um teorema ou uma contradição}
}

\newglossaryentry{deductive validity in TFL}
{
 name={validade dedutiva na LVF},
 plural={validades dedutivas na LVF},
 text={dedutivamente válido na LVF},
 description={-- um argumento é dedutivamente valido na LVF se e
somente se houver uma prova de sua conclusão a partir de suas premissas}
}

\newglossaryentry{necessary falsehood}
{
 name={falsidade necessária},
 plural={falsidades necessárias},
 description={-- uma sentença que é falsa em qualquer situação; que é impossível de ser verdadeira}
}

\newglossaryentry{necessary equivalence}
{
 name={equivalência necessária},
 plural={equivalências necessárias},
%text={necessariamente equivalente},
 description={-- propriedade de um grupo de sentenças que em qualquer situação têm sempre o mesmo valor de verdade. Ou são todas verdadeiras, ou todas falsas}
}

\newglossaryentry{sentence letter}
{
 name=letra sentencial,
 plural=letras sentenciais,
 description={-- (1) letra maiúscula possivelmente com índice numérico usada para representar na LVF uma sentença simples do português; (2) um dos elementos básicos constituintes da linguagem da LVF}
}

\newglossaryentry{atomic sentence}
{
 name=sentença atômica,
 plural=sentenças atômicas,
 description={-- expressão usada para representar uma sentença simples; na LVF corresponde a uma letra sentencial; na LPO corresponde a um símbolo de predicado seguido de nomes}
}

\newglossaryentry{symbolization key}
{
 name=chave de simbolização,
 plural=chaves de simbolização,
 description={-- (1) na LVF é uma lista que relaciona letras sentenciais com as sentenças simples do português que elas representam (simbolizam); (2) na LPO, além desta lista, contém também a definição do domínio do discurso, a relação dos indivíduos do domínio representados por cada s e a indicação da intensão (ou significado) de cada símbolo de predicado}
}

\newglossaryentry{connective}
{
 name=conectivo,
 plural=conectivos,
description={-- (1) um operador lógico da LVF usado para combinar letras sentenciais em sentenças maiores; (2) um dos elementos básicos constituintes da linguagem da LVF}
}

\newglossaryentry{negation}
{
 name=negação,
 description={-- (1) nome do conectivo `\enot', usado para simbolizar expressões do português que tenham o mesmo significado da palavra ``não''; (2) nome das sentenças cujo conectivo principal é `\enot'}
}

\newglossaryentry{conjunction}
{
 name=conjunção,
 description={-- (1) nome do conectivo `\eand', usado para simbolizar expressões do português que tenham o mesmo significado da palavra ``e''; (2) nome das sentenças cujo conectivo principal é `\eand'}
}

\newglossaryentry{conjunct}
{
 name=conjunto,
 plural=conjuntos,
 description={-- qualquer uma das duas sentenças concatenadas por `\eand' em uma conjunção}
}

\newglossaryentry{disjunction}
{
 name=disjunção,
 description={-- (1) nome do conectivo `\eor', usado para simbolizar expressões do português que tenham o mesmo significado da palavra ``ou'' em seu uso inclusivo; (2) nome das sentenças cujo conectivo principal é `\eor'}
}

\newglossaryentry{disjunct}
{
 name=disjunto,
 plural=disjuntos,
 description={-- qualquer uma das duas sentenças conectadas por `\eor' em uma disjunção}
}

\newglossaryentry{conditional}
{
 name=condicional,
 plural=condicionais,
 description={-- (1) nome do conectivo `\eif', usado para simbolizar expressões do português que significam o mesmo que ``se\ldots{}então\ldots''; (2) nome das sentenças cujo conectivo principal é `\eif'}
}

\newglossaryentry{antecedent}
{
 name=antecedente,
 plural=antededentes,
 description={-- a sentença à esquerda do símbolo `\eif' em um condicional}
}

\newglossaryentry{consequent}
{
 name=consequente,
 plural=consequentes,
 description={-- a sentença à direita do símbolo `\eif' em um condicional}
}

\newglossaryentry{biconditional}
{
 name=bicondicional,
 plural=bicondicionais,
 description={-- (1) nome do conectivo `\eiff', usado para simbolizar expressões do português que significam o mesmo que ``se e somente se''; (2) nome das sentenças cujo conectivo principal é `\eiff'}
}

\newglossaryentry{sentence of TFL}
{
 name=sentença da LVF,
 plural=sentenças da LVF,
 description={-- uma sequência de símbolos da LVF construída de acordo com as regras indutivas apresentadas na p.~\pageref{TFLsentences}}
}

\newglossaryentry{main logical operator}
{
 name=conectivo principal,
 description={-- o conectivo mais externo com relação aos parênteses em uma sentença; o último conectivo adicionado quando se constrói uma sentença de acordo com a definição indutiva}
}

\newglossaryentry{object language}
{
 name=linguagem objeto,
 description={-- é a linguagem que está sendo estudada e construída. Neste livro, as duas linguagens objetos abordadas são a LVF e a LPO}
}

\newglossaryentry{metalanguage}
{
 name=metalinguagem,
 description={-- é a linguagem que os lógicos usam para falar sobre a linguagem objeto. Neste livro, a metalinguagem é o português aumentado por certos símbolos, como as metavariáveis e certos termos técnicos, como ``válido''}
}

\newglossaryentry{metavariables}
{
 name=metavariável,
 plural=metavariáveis,
 description={-- uma variável da metalinguagem que pode representar qualquer sentença da linguagem objeto}
}

\newglossaryentry{truth value}
{
 name = valor de verdade,
 plural= valores de verdade,
 description = {-- um dos dois valores lógicos que as sentenças podem ter: o Verdadeiro e o Falso}
}

\newglossaryentry{false}
{
 name = falso,
 plural= falsos,
 text= falsa,
 description = {-- (ou \textit{falsidade}) um dos dois valores de verdade clássicos usados para classificar as sentenças; o outro é o verdadeiro. Uma sentença é classificada como falsa em uma situação quando o que ela declara não ocorre na situação}
}

\newglossaryentry{truth-functional connective}
{
 name=verofuncional,
 description={-- (ou \textit{função de verdade}) qualquer conectivo em que o valor de verdade da sentença em que ele é o conectivo principal é determinado univocamente pelos valores de verdade das sentenças componentes}
}

\newglossaryentry{valuation}
{
 name=valoração,
 description={-- (1) uma atribuição de valores de verdade a um grupo específico de letras sentenciais; (2) qualquer linha de uma tabela de verdade}
}

\newglossaryentry{truth table}
{
 name=tabela de verdade,
 description={-- uma tabela que apresenta todos os valores de verdade possíveis para uma (ou um grupo de) sentença(s) da LVF}
}

\newglossaryentry{tautology}
{
 name=tautologia,
 plural=tautologias,
 description={-- uma sentença que é verdadeira em todas as valorações; ou seja, uma sentença cuja tabela de verdade tem V em todas as linhas na coluna abaixo do conectivo principal}
}

\newglossaryentry{contradiction of TFL}
{
 name=contradição na LVF,
 text = contradição,
 description={-- uma sentença que é falsa em todas as valorações; ou seja, uma sentença cuja tabela de verdade tem F em todas as linhas na coluna abaixo do conectivo principal}
}

\newglossaryentry{equivalent}
{
 name=equivalência na LVF,
 text = equivalente,
 plural=equivalentes,
 description={-- propriedade de um grupo de sentenças que em uma tabela de verdade conjunta completa têm exatamente os mesmos valores de verdade sob a coluna do conectivo principal em cada linha da tabela. Se uma é V, todas as outras também são, e se uma é F, todas as outras também são; ou seja, a propriedade de um grupo de sentenças que em cada valoração têm valores de verdade idênticos umas aos das outras}
}

\newglossaryentry{satisfiability in TFL}
{
 name=compatibilidade na LVF,
 text=compatíveis na LVF,
 description={-- (ou \textit{satisfação conjunta na LVF}) propriedade de um grupo de sentenças da LVF cuja tabela de verdade conjunta contém pelo menos uma linha em que todas as sentenças são verdadeiras; ou seja, a propriedade de um grupo de sentenças para o qual há pelo menos uma valoração na qual todas as sentenças são verdadeiras}
}

\newglossaryentry{insatisfiability in TFL}
{
 name=incompatibilidade na LVF,
 text=incompatíveis na LVF,
 description={-- (ou \textit{insatisfação conjunta na LVF}) propriedade de um grupo de sentenças da LVF cuja tabela de verdade conjunta não contém nenhuma linha com todas as sentenças verdadeiras; ou seja, a propriedade de um grupo de sentenças para o qual não há valoração na qual todas as sentenças são verdadeiras}
}

\newglossaryentry{valid in TFL}
{
 name= validade de argumentos na LVF,
 text = válido,
 description={-- propriedade de argumentos para os quais não há linha na tabela de verdade completa do argumento na qual todas as premissas sejam verdadeiras e a conclusão seja falsa; ou seja, a propriedade de argumentos para os quais não há valorações nas quais as premissas sejam todas verdadeiras e a conclusão seja falsa}
}

% Sugiro ajustar o nome "nome", abaixo, pra não ficar muito genérico. (Ass.: Ricardo)
\newglossaryentry{name}
{
 name = nome,
 plural=nomes,
 description = {-- um símbolo da LPO usado para designar um objeto específico do domínio}
}

\newglossaryentry{predicate}
{
 name = predicado,
 plural=predicados,
 description = {-- um símbolo da LPO usado para simbolizar uma propriedade ou relação}
}


\newglossaryentry{relation}
{
 name = relação,
 plural=relações,
 description = {-- tipo especial de predicado que une (relaciona) dois ou mais indivíduos. Relações podem ser binárias, como `\blank\ ama\blank' que relaciona dois indivíduos, o que ama e o que é amado, podem ser ternárias, como `\blank pegou emprestado o \blank\ de \blank' que relaciona três indivíduos, quem pegou emprestado, a coisa que foi emprestada e quem emprestou; e também podem ser quaternárias ou, de um modo geral, $n$-árias, para qualquer $n$ finito}
}

\newglossaryentry{universal quantifier}
{
 name = quantificador universal,
 description = {-- nome do símbolo `$\forall$' da LPO usado para simbolizar uma afirmação geral: `$\forall x\, \atom{F}{x}$' é verdadeira se e somente se todos os indivíduos do domínio satisfazem o predicado $F$}
}

\newglossaryentry{variable}
{
 name = variável,
 description = {-- (1) um símbolo da LPO usado logo após os quantificadores e também como marcador de posição em fórmulas atômicas; (2) letras minúsculas, possivelmente com índices numéricos, entre $s$ e~$z$}
}

\newglossaryentry{existential quantifier}
{
 name = quantificador existencial,
 description = {-- nome do símbolo `$\exists$' da LPO usado para simbolizar uma afirmação de existência: `$\exists x\, \atom{F}{x}$' é verdadeira se e somente se pelo menos um indivíduo do domínio satisfaz o predicado~$F$}
}

\newglossaryentry{domain}
{
 name = domínio,
 description = {-- (1) o universo do discurso; (2) a totalidade assumida como alcance das variáveis e quantificadores em uma dada interpretação}
}

\newglossaryentry{empty predicate}
{
 name = {predicado vazio},
 description = {-- um predicado que não se aplica a nenhum indivíduo do domínio}
}

\newglossaryentry{term}
{
 name = termo,
 plural=termos,
 description = {-- um nome ou uma variável}
}

\newglossaryentry{formula}
{
 name = fórmula,
 plural=fórmulas,
 description = {-- uma expressão (sequência de símbolos) da LPO construída de acordo com as regras indutivas apresentadas na Seção~\S\ref{s:TermsFormulas}}
}

\newglossaryentry{scope}
{
 name = escopo,
 description = {-- (ou \textit{âmbito}, ou ainda \textit{alcance}) dado um conectivo, seu escopo são as subfórmulas da fórmula em que este conectivo é o conectivo principal}
}

\newglossaryentry{bound variable}
{
 name = variável ligada,
 plural=variáveis ligadas,
 description = {-- uma variável $\meta{x}$ é ligada em uma fórmula da LPO se todas as suas ocorrências nesta fórmula forem ligadas}
}

\newglossaryentry{ocorrencia ligada}
{
 name = ocorrência ligada,
 plural=ocorrências ligadas,
 description = {-- uma ocorrência de uma variável $\meta{x}$ é ligada se estiver dentro do escopo de um quantificador $\forall{\meta{x}}$ ou $\exists{\meta{x}}$}
}

\newglossaryentry{ocorrencia livre}
{
 name = ocorrência livre,
 plural=ocorrências livres,
 description = {-- uma ocorrência de uma variável $\meta{x}$ é livre se \textbf{não} estiver dentro do escopo de um quantificador $\forall{\meta{x}}$ ou $\exists{\meta{x}}$}
}

\newglossaryentry{free variable}
{
 name = variável livre,
 plural = variáveis livres,
 description = {-- uma variável $\meta{x}$ é livre em uma fórmula da LPO se ela tiver pelo menos uma ocorrência livre nesta fórmula}
}

\newglossaryentry{sentence of FOL}
{
 name = sentença da LPO,
 text = sentença da LPO,
 description = {-- uma fórmula da LPO que não tem variáveis livres. Uma setença também é chamada de \textit{fórmula fechada}}
}

\newglossaryentry{interpretation}
{
 name = {interpretação},
 description = {-- a especificação de um domínio do discurso, dos indivíduos deste domínio referidos pelos nomes e dos objetos (sequências de objetos) que satisfazem os predicados (relações)}
}

\newglossaryentry{substitution instance}
{
 name = instância de substituição,
 description = {-- o resultado de substituir todas as ocorrências livres de uma variável em uma fórmula por um nome}
}

\newglossaryentry{validity}
{
 name=validade lógica na LPO,
 description={-- uma sentença da LPO que é verdadeira em todas as interpretações}
}

\newglossaryentry{contradiction of FOL}
{
 name=contradição na LPO,
 text=contradição na LPO,
 description={--  uma sentença da LPO que é falsa em todas as interpretações}
} 

\newglossaryentry{valid in FOL}
{
 name=validade de argumentos na LPO,
 text = válido na LPO,
 description={-- a propriedade dos argumentos da LPO para os quais não há nenhuma interpretação em que todas as suas premissas são verdadeiras e sua conclusão é falsa. Argumentos que não são válidos na LPO, ou seja, argumentos para os quais há pelo menos uma interpretação com suas premissas verdadeiras e sua conclusão falsa são classificados com \textit{inválidos na LPO}}
 }

\newglossaryentry{support}
{
 name=sustentação na LVF,
 text = sustentam na LVF,
 description={-- as sentenças $\meta{A}_1, \ldots, \meta{A}_n$ sustentam na LVF a sentença $\meta{C}$ se não há valoração na qual $\meta{A}_1, \ldots, \meta{A}_n$ são todas verdadeiras e $\meta{C}$ é falsa}
}

\newglossaryentry{equivalent in FOL}
{
 name=equivalência na LPO,
 text = equivalente,
 plural=equivalentes,
 description={-- a propriedade de um grupo de sentenças que, em qualquer interpretação, cada sentença do grupo tem sempre o mesmo valor de verdade que todas as demais}
}

\newglossaryentry{satisfiable in FOL}
{
 name=compatibilidade na LPO,
 text=compatíveis na LPO,
 description={-- (ou \textit{satisfação conjunta na LPO}) a propriedade de um grupo de sentenças da LPO para o qual há pelo menos uma interpretação em que todas as sentenças do grupo são verdadeiras. Quando isso ocorre as sentenças deste grupo são compatíveis entre si. Por outro lado, um grupo cujas sentenças não são compatíveis entre si, ou seja, para o qual não há qualquer interpretação em que todas as sentenças do grupo sejam verdadeiras, tem a propriedade da \textit{incompatibilidade}, e suas sentenças são ditas incompatíveis entre si}
}

\newglossaryentry{disjunctive normal form}
{
 name = forma normal disjuntiva (FND),
 description = {-- está na forma normal disjuntiva uma sentença que é uma disjunção de conjunções de sentenças atômicas ou negações de sentenças atômicas}
}

\newglossaryentry{forma normal conjuntiva}
{
 name = forma normal conjuntiva (FNC),
 description = {-- está na forma normal conjuntiva uma sentença que é uma conjunção de disjunções de sentenças atômicas ou de negações de sentenças atômicas}
}

\newglossaryentry{conjunto funcionalmente completo}
{
 name = conjunto funcionalmente completo,
 description = {-- um conjunto (ou grupo) de conectivos é funcionalmente completo quando as sentenças que contém apenas conectivos deste grupo são capazes de nos dar todas as tabelas de verdade possíves}
}

\newglossaryentry{teorema}
{
 name=teorema,
 description={-- uma sentença que é provada a partir de nenhuma premissa}
}

\newglossaryentry{dedutivamente equivalente}
{
 name=dedutivamente equivalente,
 text = dedutivamente equivalente,
 plural=dedutivamente equivalentes,
 description={-- duas sentenças `$A$' e `$B$' são  dedutivamente equivalentes se e somente se cada uma puder ser provada a partir da outra}
}

\newglossaryentry{dedutivamente inconsistente}
{
 name={inconsistência dedutiva}, 
 text={dedutivamente inconsistente},
  plural={dedutivamente inconsistentes},
 description={-- a propriedade de um grupo de sentenças a partir do qual se pode provar uma contradição. Um grupo de sentenças que não é dedutivamente inconsistente, ou seja, a partir do qual não há prova de uma contradição é dito ser \textit{dedutivamente consistente}.}
}

\newglossaryentry{completude}
{
 name=completude,
 description={-- a propriedade dos sistemas lógicos nos quais $\entails $ implica $\proves$}
}

\newglossaryentry{sentence}
{
 name=sentença,
 plural=sentenças,
 description={-- (ou \textit{sentença declarativa}) frase ou expressão linguística completa, capaz de ser julgada como verdadeira ou falsa}
}

\newglossaryentry{valid}
{
 name=válido,
 plural=válidos,
 description={-- um argumento que não tem contraexemplo, no qual em qualquer situação na qual todas as suas premissas são verdadeiras, sua conclusão também é}
}

\newglossaryentry{invalid} %Alterar para argumento inválido?
{
 name=inválido,
 plural=inválidos,
 description={-- um argumento que possui contraexemplo, para o qual existe alguma situação na qual suas premissas são verdadeiras e sua conclusão é falsa}
}

\newglossaryentry{consequence}
{
 name=consequência,
 plural=consequências,
 description={-- (ou \textit{consequência lógica}) uma sentença $A$ é consequência das sentenças $B_1$, \dots, $B_n$ se e somente se não há situação em que $B_1$, \dots, $B_n$ sejam todas verdadeiras e $A$ seja falsa}
}

\newglossaryentry{counterexample}
{
 name=contraexemplo,
 plural=contraexemplos,
 description={-- uma situação que demonstra a invalidade de um argumento, na qual todas as premissas são verdadeiras, mas sua conclusão é falsa}
}

\newglossaryentry{nomologicamente valido}
{
 name=validade nomológica,
 plural=nomologicamente válidos,
 text=nomologicamente válido,
 description={-- propriedade dos argumentos para os quais não há contraexemplos que não violem alguma lei da natureza}
}

\newglossaryentry{conceitualmente valido}
{
 name=validade conceitual,
 plural=conceitualmente válidos,
 text=conceitualmente válido,
 description={-- (ou \textit{validade analítica}) propriedade dos argumentos para os quais não há contraexemplos que não violem as conexões de nossos conceitos, ou seja, o significado das palavras}
}

\newglossaryentry{formalmente valido}
{
 name=validade formal,
 plural=formalmente válidos,
 text=formalmente válido,
 description={-- propriedade dos argumentos válidos e cuja validade não depende do conteúdo de suas sentenças, mas apenas da sua forma lógica}
}

\newglossaryentry{indutivo}
{
 name=indutivo,
 plural=indutivos,
 text=argumento indutivo,
 description={-- argumento que faz uma generalização baseada em observação sobre muitas situações (passadas) e conclui sobre todas as situações (futuras)}
}


\newglossaryentry{case}
{
 name=situação,
 plural=situações,
 description={-- (ou \textit{situação possível}) um estado de coisas que representa um cenário hipotético no qual todas as sentenças de um grupo são claramente identificáveis como verdadeiras ou falsas}
}


\newglossaryentry{verdadeiro}
{
name=verdadeiro,
plural=verdadeiros,
description={-- (ou \textit{verdade}) um dos dois valores de verdade clássicos usados para classificar as sentenças; o outro é o falso. Uma sentença é classificada como verdadeira em uma situação quando o que ela declara ocorre na situação}
}


\newglossaryentry{extensao}
{
name=extensão,
description={-- a extensão de um predicado é o subconjunto do domínio do discurso que contém todos e apenas os indivíduos que satisfazem o predicado. Por exemplo, a extensão do predicado `\blank\ é honesto' no domínio dos seres humanos corresponde ao conjunto de todas as pessoas honestas}
}


\newglossaryentry{linguagem extensional}
{
name=linguagem extensional,
description={-- é uma linguagem na qual quaisquer dois predicados são indistinguíveis em todas  as interpretações nas quais têm a mesma  extensão. Em oposição às linguagens extensionais, existem as linguagens intensionais, nas quais dois predicados com a mesma extensão em uma dada interpretação podem ainda ser distinguidos quando têm significados (intensões) diferentes. A LPO é uma linguagem extensional, enquanto o português é intensional. Por exemplo, usualmente interpretados, os predicados `\blank\ é um animal racional' e `\blank\ é um primata com polegar opositor' têm supostamente a mesma extensão, o conjunto dos seres humanos, e por isso, suas simbolizações na LPO serão indistinguíveis nesta interpretação usual. Mas estes predicados são claramente distinguíveis em português, porque ainda que o conjunto dos seres racionais e o dos primatas com polegar opositor sejam idênticos, o significado de  `ser racional' é bem diferente do de `ter polegar opositor'}
}


\newglossaryentry{satisfacao}
{
 name=satisfação,
 plural=satisfaz,
 text=satisfaz,
 description={-- Um indivíduo do domínio de uma interpretação satisfaz uma fórmula (sentença com variável livre) quando a sentença obtida desta fórmula através da substituição de sua variável livre pelo nome do indivíduo é verdadeira na interpretação. Por exemplo, na interpretação dada pelo mundo real, a fórmula `x foi o primeiro reitor da UFRN' é satisfeita pelo indivíduo cujo nome é `Onofre Lopes da Silva', porque a sentença `Onofre Lopes da Silva foi o primeiro reitor da UFRN' é verdadeira. Definindo de um modo mais formal: seja $\mathbf{I}$ uma interpretação e $i$ o indivíduo do domínio que, de acordo com $\mathbf{I}$, é nomeado por `$d$'. Dizemos que $i$ satisfaz a fórmula $\meta{A}(\meta{x})$ em $\mathbf{I}$, quando a sentença $\meta{A}(\meta{d})$, obtida de $\meta{A}(\meta{x})$ pela substituição de todas as ocorrências livres de `$x$' pelo nome `$d$' for verdadeira em $\mathbf{I}$}
}

\newglossaryentry{literal}
{
name=literal,
text=literais,
description={-- chamamos de literal qualquer sentença da LVF que seja ou uma letra sentencial (sentença atômica) ou a negação de uma letra sentencial. Por exemplo, são literais as sentenças: $A, \enot B, A_3, \enot F_2, ...$}
}