\chapter{Prefácio}

O principal objetivo da lógica é avaliar argumentos e separar os bons, chamados válidos, dos maus, os inválidos.
Um argumento é um grupo de sentenças que corresponde à expressão linguística de um raciocínio.
Uma das sentenças, a conclusão, é o ponto final do raciocínio, aquilo que se conclui.
As outras, as premissas, são o ponto inicial, as informações das quais se parte para chegar à conclusão.
A ideia é que os argumentos bons do ponto de vista lógico, os válidos, são aqueles que expressam raciocínios infalíveis; raciocínios em que a conclusão a que se chega está infalivelmente justificada a partir das premissas de que se parte.
A lógica enquanto disciplina busca caracterizar os raciocínios infalíveis através de uma relação formal entre as sentenças dos argumentos que os expressam: a consequência lógica.
%Esta relação não depende do assunto sobre o qual as sentenças tratam e, ao vigorar entre as premissas e a conclusão de todo argumento válido, realiza o objetivo da lógica de identificar os raciocínios infalíveis.
Esta relação não depende do assunto sobre o qual as sentenças tratam e vigora entre as premissas e a conclusão de todo argumento válido.

Vamos, no decorrer deste livro, examinar detidamente a relação de consequência e aprender alguns dos principais métodos formais empregados em sua abordagem.
No entanto, para você ter uma primeira ideia de onde está pisando ao iniciar seus estudos, vejamos aqui um exemplo ilustrativo.
Suponha que Virgínia more em Ponta Negra, mas que você não saiba disso.
De que modo você poderia ficar sabendo?
Bem, você pode ficar sabendo que Virgínia mora em Ponta Negra diretamente, caso você venha a conhecer a casa de Virgínia, ou seu endereço, pelo menos.
Mas você pode também ficar sabendo indiretamente, através da lógica.
Suponha que você ouça trechos de uma conversa de Virgínia com uma amiga, e que tenha ouvido as seguintes sentenças ditas por ela:
\begin{earg}
	\item Eu vou à praia de Ponta Negra a pé todas as manhãs.
	\item Eu permaneço 30 minutos na praia todos os dias.
	\item Eu costumo sair de casa para a praia às 6h15 e às 7h já estou de volta.
\end{earg}
Bem, você não ouviu Virgínia dizendo onde morava, mas se você conhece Natal, você sabe que:

\begin{earg}
	\item[4.] Não é possível ir e voltar a pé à praia de Ponta Negra de nenhum outro bairro (que não seja Ponta Negra) em 15 minutos ou menos.
\end{earg}
Então, se você juntar o que ouviu Virgínia dizer (1, 2 e 3) com o que sabe sobre Natal (4), você conclui que:
\begin{earg}
	\item[5.]  Virgínia mora em Ponta Negra.
\end{earg}
Em outras palavras, a afirmação de que Virgínia mora em Ponta Negra é consequência das afirmações 1, 2, 3 e 4.
É exatamente este tipo de relação de consequência entre afirmações que é o assunto da lógica.

Estudamos lógica, principalmente, para saber identificar e caracterizar os casos em que a verdade de certas sentenças, tais como as sentenças 1 a 4 acima, configura-se em uma justificativa suficiente para a verdade de alguma outra sentença, tal como a sentença 5.
Dizer, então, que uma sentença é consequência lógica de outras, é dizer que ela não pode ser falsa quando essas outras são verdadeiras.
Não é possível que Virgínia não more em Ponta Negra se as sentenças 1 a 4 forem todas verdadeiras.\footnote{
	Contemporaneamente as pesquisas em lógica ampliaram-se e é possível encontrar abordagens heterodoxas onde são estudados outros tipos de relação de consequência.
	Neste livro, no entanto, nos limitaremos à concepção ortodoxa de consequência lógica como preservação da verdade.}

A relação de consequência é fundamental porque ela nos ajuda a reconhecer a verdade.
Entendendo-a melhor, poderemos identificar situações que inevitavelmente ocorreriam sempre que certas outras situações tivessem ocorrido.
Além do reconhecimento da verdade, a lógica nos ajuda também a reconhecer a mentira (ou a falsidade).
A compatibilidade, uma outra noção importante que estudaremos, nos ajudará a identificar se as sentenças de um dado grupo podem ou não ser todas conjuntamente verdadeiras.
Se elas não podem, qualquer um que as tenha afirmado todas ou está mentindo ou está equivocado.
%Também consideraremos a relação de satisfatoriedade conjunta entre sentenças, ou seja,
%também faz parte dos interesses da lógica saber identificar se as sentenças de um dado grupo são ou não mutuamente contraditórias; 


%além de algumas outras noções relacionadas.
%Os conceitos que fazem parte da lógica podem ser definidos tanto de modo semântico, através de definições precisas de consequência, baseadas em interpretações da linguagem, quanto de modo demonstrativo, através de sistemas formais de manipulação simbólica.

A lógica formal ocupa posição privilegiada na metodologia da filosofia. 
As suposições especulativas, explicações e sistemas que os filósofos propõem  são, quase sempre, abstratos demais para que possam ser avaliados diretamente. 
Sua profundidade, relevância, veracidade, consistência e plausibilidade são avaliadas e comparadas principalmente através das suas consequências lógicas que nos são mais facilmente acessíveis.
Um recurso comum dos diálogos platônicos, por exemplo, é a proposição especulativa de algum princípio moral ou metafísico abstrato que posteriormente é descartado quando se demonstra que sua admissão tem consequências (lógicas) mais facilmente percebidas como inaceitáveis.

Mas a lógica não é importante para a filosofia apenas como instrumento metodológico.
Ela, tanto quanto a ética, a metafísica, a política, a estética e a epistemologia é também uma disciplina filosófica.
A lógica é parte da filosofia porque seu objeto de estudo, a relação de consequência, levanta questões cujas respostas não podem ser dadas através de pesquisa empírica.
Os princípios lógicos, tanto quanto os sistemas éticos, metafísicos e das demais áreas da filosofia, são resultado de especulação racional livre e são objeto de divergências e disputas para as quais não há um tribunal último capaz de resolvê-las objetivamente.
A lógica faz parte da filosofia porque como todas as outras disciplinas filosóficas ela lida com questões cujas respostas sempre envolverão algum tipo de escolha, de engajamento.

Este caráter mais filosófico da lógica costuma ser cuidadosamente apagado e escondido na esmagadora maioria dos livros didáticos, que tendem a apresentar uma versão higienizada da lógica, imitando o que costumeiramente é feito no ensino das ciências e da matemática.

Talvez a justificativa dessa escolha pedagógica seja a disputável ideia de que é preciso  primeiro aprender os rudimentos de um assunto (a lógica), para depois questioná-lo e situá-lo filosoficamente.
Não cabe discutir aqui os defeitos e méritos desta abordagem. 
No entanto, o que nossa experiência ao longo dos anos com o ensino da lógica para estudantes de filosofia nos sugere é que os alunos e alunas naturalmente se interessam pelas questões filosóficas que a lógica suscita e tendem a aproveitar melhor a disciplina quando são introduzidos simultaneamente à lógica e à filosofia da lógica.

O livro \forallx: \emph{Calgary}, base deste, foi o primeiro manual de lógica com o qual nos deparamos que não esconde nem evita as dificuldades filosóficas da lógica, e a apresenta de um modo que nos parece melhor tanto para ajudar o estudante que se especializará em outra área da filosofia a entender o caráter da lógica como disciplina filosófica, quanto para preparar a estudante que se especializará em lógica a compreender o ambiente plural das pesquisas contemporâneas, no qual não há mais nenhum protagonismo para a lógica clássica, e onde o debate e a divergência são tão ou mais intensos que nas outras áreas da filosofia.

A principal diretriz que guiou esta nossa adaptação do livro \forallx: \emph{Calgary} foi uma tentativa de aprofundar o tratamento dos pontos de filosofia da lógica, juntamente com adaptações contextuais à realidade de nossos estudantes.


%Dado que um elemento essencial da atividade filosófica, que é base racional de sua metodologia, é o relacionamento entre suposições especulativas com as diversas consequências destas suposições, a lógica formal é, obviamente, uma sub-disciplina central da filosofia.
%É com base nas consequências das definições, suposições e especulações que propõem, que os filósofos avaliam a profundidade, relevância, veracidade, consistência e plausibilidade de suas propostas.
Mas a lógica formal não é importante só para a filosofia.
Ela também tem papel privilegiado tanto na matemática quanto na ciência da computação.
Em matemática, as linguagens formais são usadas para descrever não estados de coisas ``do dia a dia'', mas estados de coisas matemáticos.
Os matemáticos também se interessam pelas consequências de definições e suposições e, para eles, também é importante estabelecer essas consequências (que eles chamam de ``teoremas'') usando métodos completamente precisos e rigorosos.
A lógica formal fornece esses métodos.
Na ciência da computação, a lógica formal é aplicada para descrever o estado e o comportamento dos sistemas computacionais, tais como circuitos, programas, bancos de dados, etc.
Os métodos da lógica formal também podem ser usados para estabelecer as consequências de tais descrições, como, por exemplo, se um dado circuito está ou não livre de erros, ou se um programa faz o que se pretende que ele faça, ou se um banco de dados é consistente, ou se algo é verdadeiro a partir dos dados nele contidos.

Este livro está dividido em nove partes.
A parte~\ref{ch.intro} introduz o assunto e as noções da lógica de maneira informal, ainda sem utilizar uma linguagem formal.
As partes \ref{ch.TFL}, \ref{ch.TruthTables} e \ref{ch.NDTFL} tratam da linguagem verofuncional.  Nesta linguagem as sentenças são formadas a partir de sentenças básicas através de certos termos (`ou', `e', `não', `se \dots então') que conectam sentenças mais simples de modo a formar outras sentenças mais complexas.
Noções lógicas tais como a relação de consequência são discutidas de duas maneiras:
semanticamente, usando o método das tabelas de verdade (na Parte~\ref{ch.TruthTables}) e demonstrativamente, usando um sistema de derivações formais (na Parte~\ref{ch.NDTFL}).
As partes \ref{ch.FOL}, \ref{ch.semantics} e \ref{ch.NDFOL} lidam com uma linguagem mais complexa, a da lógica de primeira ordem.
Além dos conectivos da lógica verofuncional, esta linguagem inclui também nomes, predicados, a relação de identidade e os quantificadores.
Esses elementos adicionais da linguagem a tornam muito mais expressiva do que a linguagem verofuncional, e passaremos um bom tempo investigando quanto se pode expressar nela.
As noções da lógica de primeira ordem também são definidas tanto semanticamente, através interpretações (na Parte \ref{ch.semantics}), quanto demonstrativamente (na Parte \ref{ch.NDFOL}), usando uma versão mais complexa do sistema de derivação formal introduzido na Parte~\ref{ch.NDTFL}.
%\ref{ch.ML}
A Parte VIII discute uma extensão da LVF (a lógica verofuncional) obtida a partir de operadores não verofuncionais para a possibilidade e a necessidade, conhecida como lógica modal.
%\ref{ch.normalform}
A Parte IX abrange dois tópicos avançados:
o tópico das formas normais (conjuntiva e disjuntiva) e da adequação expressiva dos conectivos verofuncionais, e o tópico da correção do sistema de dedução natural para a LVF.

Nos apêndices, você encontrará uma discussão sobre notações alternativas para as linguagens tratadas neste texto, uma outra sobre sistemas de derivação alternativos, além  de um guia de referência rápida listando a maioria das regras e definições importantes.
Os termos principais estão listados em um glossário no final.

Este livro  é uma versão adaptada e ampliada do livro \forallx: \emph{Calgary}, que é uma versão revista e ampliada por Aaron Thomas-Bolduc e Richard Zach do livro \forallx: \emph{Cambridge}, que, por sua vez, é uma versão revista e ampliada por Tim Button do livro \forallx, de P.D. Magnus.
Além disso, a editoração gráfica desta edição se baseia na estrutura e digramação de Mark Lyall para a versão de Thomas-Bolduc e Zach, que está livremente disponível em \hbox{\href{https://forallx.openlogicproject.org}{forallx.openlogicproject.org}}.

Este livro é mesmo Para Tod$x$s (para todas e para todos).
Você é livre para copiar e redistribuir gratuitamente este material em qualquer meio ou formato, remixar, transformar e desenvolvê-lo para qualquer finalidade, mesmo comercialmente, desde que respeite as restrições da licença  \href{https://creativecommons.org/licenses/by/4.0/}{Creative Commons Attribution 4.0} descritas na página \pageref{cc4by}.


