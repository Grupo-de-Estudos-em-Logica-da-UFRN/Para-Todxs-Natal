%!TEX root = forallxyyc.tex
\thispagestyle{empty}
\onecolumn
\ 
\vfill

\parbox{3 in}{

Este é um livro de autoria coletiva, distribuição livre e edição dinâmica que pode, eventualmente, sofrer acréscimos, alterações e correções.
Esta é a edição de \mydate.
Você pode verificar neste link
\href{https://github.com/Grupo-de-Estudos-em-Logica-da-UFRN/Para-Todxs-Natal/blob/main/paratodxsnatal.pdf}{$\langle$última versão$\rangle$}
se alguma versão mais recente já está disponível.
Você pode também colaborar com o aprimoramento deste livro e tornar-se coautor ou coautora das próximas edições.
Caso encontre erros ou tenha sugestões de reescrita ou críticas, você pode enviá-los através deste formulário 
\href{https://forms.gle/yd4yH9WAo6TxAiSj8}{$\langle$correções e sugestões$\rangle$}.
Se alguma de suas sugestões for incorporada, ainda que seja apenas uma vírgula, seu nome será incluído na lista de coautores colaboradores. 

%Esta é a versão rascunho (0.8 -- \mydate) de um livro que ainda não está pronto.
%Você pode verificar neste link
%\href{https://github.com/Grupo-de-Estudos-em-Logica-da-UFRN/Para-Todxs-Natal/blob/main/ %paratodxsnatal.pdf}{$\langle$última versão$\rangle$}
%\hbox{\url{https://rb.gy/w3ovdf}}
%se alguma versão mais recente já está disponível.
%Este livro está sendo testado em algumas turmas de uma disciplina introdutória de lógica da %UFRN e planejamos finalizar uma primeira edição para publicação ainda em 2021. 
%A versão atual já contempla todo o conteúdo pretendido, mas o livro, de autoria coletiva, está em processo de revisão e ainda sofre com certa falta de padronização de notação e nomenclatura, falta de homogeneização do tom, além de muitos erros de escrita e digitação.
%Você pode nos ajudar na revisão e fazer parte do coletivo de coautores e coautoras reportando erros, críticas e sugestões neste formulário
%\href{https://forms.gle/yd4yH9WAo6TxAiSj8}{$\langle$correções e sugestões$\rangle$}
%\hbox{\url{https://forms.gle/yd4yH9WAo6TxAiSj8}} 
}

\vfill

\parbox{3 in}{
%P.D.\ Magnus is an associate professor of philosophy in Albany, New York. His primary research is in the philosophy of science.

%\
%\\Tim Button is a University Lecturer, and Fellow of St John's College, at the University of Cambridge. His first book, \emph{The Limits of Realism}, was published by Oxford University Press in 2013.
}
\vfill
