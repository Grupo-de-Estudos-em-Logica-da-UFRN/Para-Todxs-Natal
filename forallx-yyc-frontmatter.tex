%!TEX root = forallxyyc.tex

% Bastard Title

\pagestyle{empty}

\vspace*{80pt}

\begin{raggedleft}
\fontsize{30pt}{24pt}\sffamily
\selectfont
  \textbf{Para Tod{\fontsize{37pt}{24pt}\selectfont\rmfamily\textit{x}}s:
  Natal}

\medskip\fontsize{18pt}{20pt}\selectfont

\textbf{uma introdução à\\ lógica formal}


%\begin{raggedleft}
%\fontsize{30pt}{24pt}\sffamily
%\selectfont
%  \textbf{ParaTod{\HUGE \emph{x}}s 
%  {\fontsize{37pt}{24pt}\selectfont\rmfamily \ --\ } 
%  Natal}

%\medskip\fontsize{18pt}{20pt}\selectfont

%\textbf{uma introdução à\\ lógica formal}

\vfill
\fontsize{12pt}{16pt}\selectfont \textit{De } \textbf{P.~D. Magnus}\\
\textbf{Tim Button}\\
\textit{com acréscimos de}\\
\textbf{J.~Robert Loftis}\\
\textbf{Robert Trueman}\\
\textit{remixado e revisado por}\\
\textbf{Aaron Thomas-Bolduc}\\ \textbf{Richard Zach}\\
\textit{adaptado, re-remixado, re-revisado e ampliado  pelo}\\ \textbf{GEL - Grupo de Estudos em Lógica da UFRN}

\vfill
%\textbf{maio, 2020}\par
\textbf{\today}\par
\end{raggedleft}


\newpage

\thispagestyle{empty}
\onecolumn
\ 
\vfill

\parbox{3 in}{
Esta é a versão rascunho (0.4) de um livro que ainda não está pronto.
Você pode verificar neste link, \hbox{http://tiny.cc/wwpksz}, se alguma versão mais recente já está disponível.
Este livro está sendo testado em uma disciplina de lógica da UFRN e pretendemos finalizar uma primeira edição para publicação no final de 2020 ou início de 2021. 
%Suas versões de rasEsta versão (0.3) está sendo testada com estudantes da UFRN e é a base para uma versão final que pretende-se seja devidamente publicada no final de 2020 ou início de 2021.
Para esta finalização ainda falta produzirmos duas partes do livro, uma tratando de rudimentos de lógica modal e outra de metateoria, além de considerações finais, alguns apêndices sobre notação e jargão e um glossário.
Falta, principalmente, uma rigorosa revisão e homogeneização do texto.
Caso você tenha interesse neste projeto e queira participar mais ativamente, conctacte-nos através do email durante10@gmail.com. 
Também agradecemos se nos contactar apontando  erros (sim, ainda há muitos!), ou tenha sugestões, discordâncias, críticas,... suas contribuições serão todas muito bem-vindas.

%\medskip

%The same might be said for this volume, although readers are forgiven if they take a break for snacks after \emph{two} readings.
}

\newpage

\noindent Esta obra é baseada no livro \forallx: \textit{Calgary} de P.D. Magnus, Tim Button, J. Robert Loftis, Robert Trueman, Aaron Thomas-Bolduc e Richard Zach, que foi utilizado aqui sob a licença \href{https://creativecommons.org/licenses/by/4.0/}{CC BY 4.0}.
Mas para você entender direito a autoria deste livro, é preciso listar alguns outros livros e explicar a relação entre todos eles.

\begin{enumerate}
   \item \forallx, de P.D. Magnus

   \item \forallx: \textit{Cambridge}, de Tim Button

   \item \forallx: \textit{Calgary} de P.D. Magnus, Tim Button, J. Robert Loftis, Robert Trueman, Aaron Thomas-Bolduc e Richard Zach
   
   \item \textit{Metatheory}, de Tim Button
   
   \item  \forallx: \textit{Lorain Conty Remix}, de Cathal Woods e J. Robert Loftis
   
   \item \textit{A Modal Logic Primer}, de Robert Trueman
\end{enumerate}

\noindent A história é a seguinte.
P. D. Magnus escreveu o livro (1). Tim Button produziu o livro (2) baseado no livro (1) e Aaron Thomas-Bolduc junto com Richard Zach produziram  livro (3) com base no livro (2).
Mas Thomas-Bolduc e Zach também utilizaram materiais dos livros (1), (4), (5) e (6) na produção do livro (3).
Aí nós, do GEL-UFRN (grupo de estudos em lógica do departamento de filosofia da universidade federal do Rio Grande do Norte), 
utilizamos o livro (3) como texto base para a produção deste livro, que não é uma tradução, mas uma adaptação livre, em que fizemos modificações, alterações e inclusões.
Nossa adaptação foi feita tendo estudantes de graduação em filosofia como público alvo.
Nossa experiência ao longo dos anos nos sugere que  os estudantes de filosofia se interessam mais e aproveitam mais a lógica quando a estudam juntamente com a filosofia da lógica.
O que fizemos, então, foi amplificar os elementos de filosofia da lógica que já estavam presentes no livro (3), e adaptá-los às nossas concepções.

Os livros (1), (2), (3) e (4) estão todos sob a licença \href{https://creativecommons.org/licenses/by/4.0/}{CC BY 4.0}, e os livros (5) e (6) foram utilizados por Thomas-Bolduc e Zach com permissão.



%\bigskip

Esta obra está protegida sob a licença \href{https://creativecommons.org/licenses/by/4.0/}{Creative Commons \hbox{Attribution 4.0}}. 
Você é livre para copiar e redistribuir este material em qualquer meio ou formato, remixar, transformar e desenvolvê-lo para qualquer finalidade, mesmo comercialmente, nos seguintes termos:
\begin{itemize}
\item Você deve dar o crédito apropriado, fornecer um link para a licença e indicar se foram feitas alterações. Você pode fazê-lo de qualquer maneira razoável, mas não de maneira que sugira que os licenciantes (os demais autores) endossam você ou seu uso.
\item Você não pode aplicar termos legais ou medidas tecnológicas que restrinjam legalmente outras pessoas a fazer o que a licença permite.
\end{itemize}

\noindent A editoração gráfica deste livro foi produzida com base no código fonte \LaTeX{} do livro (3), que está disponível em \hbox{\href{https://forallx.openlogicproject.org}{forallx.openlogicproject.org}}.
A capa e o design são de Mark Lyall.
%Esta versão é a revisão \gitAbbrevHash{} (\gitAuthorDate).
Esta versão é a revisão (0.4) (\today)

%\bigskip
%\noindent A preparação da versão em inglês deste livro foi possível graças a uma bolsa concedida por \href{https://www.ucalgary.ca/taylorinstitute/}{Taylor Institute for Teaching and Learning}.

%\bigskip
%\noindent
%\href{https://www.ucalgary.ca/taylorinstitute/}{\includegraphics[width=8cm]{assets/ti-color}}

%\bigskip
%\noindent Capa e design de Mark Lyall.